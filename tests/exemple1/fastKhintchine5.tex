\documentclass[a4paper,11pt]{amsart}
\usepackage{amsfonts,amsthm,amsmath,amssymb,graphicx}
\theoremstyle{plain}
\newtheorem{lem}{Lemma}[section]
\newtheorem{prop}[lem]{Proposition}
\newtheorem{thm}[lem]{Theorem}
\newtheorem{cor}[lem]{Corollary}
\theoremstyle{definition}
\newtheorem{defn}[lem]{Definition}
\newtheorem{ex}[lem]{Example}
\theoremstyle{remark}
\newtheorem*{rem}{Remark}
\newtheorem*{notat}{Notation}
\numberwithin{equation}{section}
\DeclareMathOperator{\cl}{cl}
\DeclareMathOperator{\diam}{diam}
\DeclareMathOperator{\vol}{vol}
\DeclareMathOperator{\dist}{dist}
\DeclareMathOperator{\var}{var}
\newcommand{\C}{\mathbb C}
\newcommand{\clC}{\widehat \C}
\newcommand{\R}{\mathbb R}
\newcommand{\Z}{\mathbb Z}
\newcommand{\N}{\mathbb N}
\newcommand{\Q}{\mathbb Q}
\newcommand{\al}{\alpha}
\newcommand{\bd}{\partial}
\renewcommand{\epsilon}{\varepsilon}

\usepackage[usenames]{color}
\title[Fast Khintchine spectra]{Upper and lower fast Khintchine spectra in
continued fractions}
\date{}

\author{Lingmin Liao}
\address{Lingmin Liao\\LAMA UMR 8050, CNRS,
Universit\'e Paris-Est Cr\'eteil, 61 Avenue du
G\'en\'eral de Gaulle, 94010 Cr\'eteil Cedex, France}
\email{lingmin.liao@u-pec.fr}
\author{Micha\l\ Rams}
\address{Micha\l\ Rams\\Institute of Mathematics\\ Polish
Academy of Sciences\\ ul.
\'Sniadeckich 8, 00-956 Warszawa\\ Poland }
\email{rams@impan.pl}
%\thanks{M.R. was partially supported by the MNiSW grant N201
%607640 (Poland).}
%
%
%
%\thanks{L.L. was partially supported by 12R03191A
%- MUTADIS (France).} %A part of this paper was written during the visit of L.L. to the NCTS in Taiwan.}



\begin{document}
\begin{abstract}
Every $x\in [0,1)$ can be expanded as a continued fraction: $x=[a_1(x), a_2(x),\cdots]$. Let $\psi : \mathbb{N} \rightarrow \mathbb{N}$ be a function with $\psi(n)/n\to \infty$ as $n\to
\infty$. The (upper, lower) fast Khintchine spectrum for $\psi$ is defined as the Hausdorff dimension of the set of numbers $x\in (0,1)$ for which the (upper, lower) limit of $\frac{1}{\psi(n)}\sum_{j=1}^n\log a_j(x)$ is equal to $1$. The fast Khintchine spectrum was determined by Fan, Liao, Wang, and Wu.
We calculate the upper and lower fast Khintchine spectra. These three spectra can be different.
\end{abstract}

\maketitle
\def\thefootnote{}
\footnote{2010 {\it Mathematics Subject Classification}: Primary 11K50 Secondary 37E05, 28A78}
\def\thefootnote{\arabic{footnote}}


\section{Introduction}
Each irrational
number $x\in [0,1)$ admits a unique infinite continued fraction
expansion of the form
\begin{eqnarray}\label{ff1}
x=\frac{\displaystyle 1}{\displaystyle a_1(x)+ \frac{\displaystyle
1}{\displaystyle a_2(x)+\frac{\displaystyle 1}{\displaystyle
a_3(x)+\ddots}}},
\end{eqnarray}
where the integers $a_n(x)$, called the partial quotients of $x$, can be generated by using the Gauss
transformation $T:[0,1)\to [0,1)$ defined by  $$ T(0):=0, \
T(x)=\frac{1}{x} \ {\rm{(mod \ 1)}}, \ {\rm{for}}\ x\in (0,1).
$$ 
In fact, let $a_1(x)= \lfloor x^{-1}\rfloor$ ($\lfloor \cdot \rfloor $ stands for the integral part), then $a_n(x)=a_1(T^{n-1}(x))$ for $n\ge 2$. For simplicity,  (\ref{ff1}) is often written as $x=[a_1,a_2,\cdots]$.  
%The $n$-th convergent $p_n(x)/q_n(x)=[a_1,\cdots, a_n]$ of $x$
%is the finite truncation of (\ref{ff1}).

For any $x\in (0,1)$, the Khintchine exponent of $x$ is defined by the limit (if it exists)
\[
\xi(x):=\lim_{n\to\infty}\frac{\log
a_1(x)+\cdots+\log a_n(x)}{n}.
\]
Khintchine \cite{Kh} proved that for Lebesgue almost all points $x$, we have 
\[
\xi(x)=\int_0^1 {\log a_1(x) \over (1+x)\log 2} dx=2.6854....
\]

Let $\psi:\mathbb{N}\to \mathbb{N}$ and let $\alpha>0$. Define $$
E(\psi, \alpha)=\left\{x\in [0,1): \lim_{n\to\infty}\frac{\log
a_1(x)+\cdots+\log a_n(x)}{\psi(n)}=\alpha\right\}.
$$ When $\psi(n)=n$, the set $E(\psi, \alpha)$ is a level set of the
Khintchine exponent, whose Hausdorff dimension is determined in \cite{FLWW}.  The function of the Hausdorff dimension associated to each $\alpha$ is called the {\it Khintchine spectrum}.
 
Later, in \cite{FLWW13}, the authors studied the fast Khintchine spectrum, i.e. the Hausdorff dimension of $E(\psi, \alpha)$
where $\psi$ satisfies that $\psi(n)/n\to \infty$ as $n\to\infty$. In this case, it turns out that the Hausdorff dimension does not depend on the level $\alpha$, but only on the increasing rate of $\psi$. More precisely, let $\psi$ and $\tilde{\psi}$ be two functions defined on
$\N$. We say $\psi$ and $\tilde{\psi}$ are {\em equivalent} if $\frac{\psi(n)}{\tilde{\psi}(n)}\to 1$ as $n\to \infty$. We denote the Hausdorff dimension by $\dim_H$. The authors of \cite{FLWW13} proved the following theorem.
\begin{thm}[\cite{FLWW13}]\label{th-FLWW}  Let $\psi:\N\to \N$ with ${\psi(n)}/{n}\to \infty$ as $n\to \infty$.
If $\psi$ is equivalent to an increasing function, then for all $\alpha>0$, $E(\psi)\ne
\emptyset$ and
$$ \dim_HE(\psi, \alpha)=\frac{1}{1+\beta}, \ \ {\text{with}} \ \beta=\limsup_{n\to
\infty}\frac{\psi(n+1)}{\psi(n)}.
 $$ Otherwise,
 $E(\psi, \alpha)=\emptyset$ for all $\alpha>0$.
\end{thm}
When the sets $E(\psi, \alpha)$ are not empty, the dimensional function associated to $\psi$ (and $\alpha$) is called the {\it fast Khintchine spectrum} in \cite{FLWW13}.  

In this note, we consider the following sets
$$
\overline{E}(\psi)=\left\{x\in [0,1]: \limsup_{n\to\infty}\frac{\log a_1(x)+\cdots+\log a_n(x)}{\psi(n)}=1\right\},
$$
and 
$$
\underline{E}(\psi)=\left\{x\in [0,1]: \liminf_{n\to\infty}\frac{\log a_1(x)+\cdots+\log a_n(x)}{\psi(n)}=1\right\}.
$$
Their Hausdorff dimensions are called {\it upper} and {\it lower fast Khintchine spectra}.
 
Remark that we only consider the level $\alpha=1$ here, since for other levels the Hausdorff dimension will not change, as in Theorem \ref{th-FLWW}.
\begin{thm}\label{main} Assume that $\psi:\mathbb{N}\to\mathbb{N}$
satisfy $\psi(n)/n\to\infty$ as $n\to\infty$. Write
$$\liminf_{n\to\infty}\frac{\log \psi(n)}{n}=\log b \quad \text{and} \quad \limsup_{n\to\infty}\frac{\log \psi(n)}{n}=\log B.$$ 
Assume $b, B \in (1, \infty]$. Then % the sets $\overline{E}(\psi)$ and $\underline{E}(\psi)$ are always nonempty and
$$
\dim_H\overline{E}(\psi)=\frac{1}{1+b}\quad \text{and} \quad \dim_H\underline{E}(\psi)=\frac{1}{1+B}.
$$
\end{thm}
We remark that the three values $\beta, b$ and $B$ are in general different even though we always have the relation $b\leq B \leq \beta$.  We also remark that the set 
$\overline{E}(\psi)$ and $\underline{E}(\psi)$ are always nonempty.

%Our results show that the Gauss dynamical system is different to the classical compact continuous dynamical systems. 
%
%Let $(X,d)$ be a compact metric space and $T$ is a continuous transformation on $X$. 
%Let $\varphi : X \rightarrow \mathbb{R}$ be a real-valued continuous function. Denote  
%$$S_n\varphi(x):=\sum_{j=0}^{n-1}\varphi(T^jx) \quad \text{and} \quad \overline{\varphi}(x):=\lim_{n\to\infty}{1 \over n} S_n\varphi(x).$$ For $\alpha \in [\min  \overline{\varphi}, \ \max \overline{\varphi}]$, we consider the set
%\[
%F_\varphi (\alpha):=\left\{x\in X: \lim_{n\to \infty}{S_n \varphi(x) \over n} =\alpha\right\},
%\]
%and the sets $\overline{F}_\varphi (\alpha)$ and $\underline{F}_\varphi (\alpha)$ defined by replacing the limit by limsup and limsup.
%%A result of Barreira and Schmeling \cite{} implies 
%
%Here we need not to study the fast increasing rate of the Birkhoff sum since the function $\varphi$ is bounded as a continuous function on a compact space. However, for the linear increasing rate, we have the following propostition.
%
%It is well known that in the case of a H\"older potential on a smooth conformal repeller for any $\alpha$ the Hausdorff dimensions  of the above three sets are equal.
%\begin{prop} For all $\alpha \in [\min  \overline{\varphi}, \ \max \overline{\varphi}]$, we have
%\[
%\dim_H(F_\varphi (\alpha))=\dim_H(\overline{F}_\varphi (\alpha))=\dim_H(\underline{F}_\varphi (\alpha)).
%\]
%\end{prop}
%We could not find a proof of this folklore result in the literature, so we will give a short proof in the appendix.

%The situation is different for the case of dynamical systems with infinite many branches. 





\bigskip
\section{Preliminary}
%Recall that for any irrational number $x\in [0,1)$, $p_n(x)$ and
%$q_n(x)$ are the numerator and denominator of  the $n$-th convergent
%of $x$. % denoted by $\frac{p_n}{q_n}$ of $x$ is given by
%%$$ \frac{p_n}{q_n}=\frac{p_n(x)}{q_n(x)}=\frac{\displaystyle
%%1}{\displaystyle a_1(x)+ \frac{\displaystyle 1}{\displaystyle
%%\ddots+\frac{\displaystyle 1}{\displaystyle a_n(x)}}}.
%%$$
%It is known that $p_n=p_n(x)$ and $q_n=q_n(x)$ can be obtained recursively by the following
%relations.
%\begin{equation}\label{ff2.1}
%\begin{split}
%p_n=a_n(x) p_{n-1}+p_{n-2}, \ \ q_n=a_n(x)
%q_{n-1}+q_{n-2}
%\end{split}
%\end{equation}with the conventions $p_0=q_{-1}=0$ and
%$p_{-1}=q_0=1$. For each $n\ge 1$,
%\begin{equation}\label{7}
% p_{n-1}q_n-p_nq_{n-1}=(-1)^n.
%\end{equation}
%
 For any $n \geq 1$ and $(a_1,a_2,\cdots,a_n)\in
\mathbb{N}^n$, define $$ I_n(a_1, a_2, \cdots, a_n)=\big\{x\in
[0,1):\ a_1(x)=a_1, \cdots, a_n(x)=a_n\big\},
$$ which is the set of numbers starting with $(a_1,\cdots, a_n)$ in their continued fraction expansions,
and is called a {\em basic interval} of order $n$. The length of a basic interval will be denoted by $|I_n|$.
%
%Note that $p_n$ and $q_n$ are determined  by the first $n$ partial
%quotients of $x$. So all points in $I_n(a_1,\cdots, a_n)$  determine
%the same $p_n$ and $q_n$. Hence sometimes, we write
%$p_n=p_n(a_1,\cdots, a_n)$ and $q_n=q_n(a_1,\cdots, a_n)$ to denote
%$p_n(x)$ and $q_n(x)$ for $x\in I_n(a_1,\cdots, a_n)$.
\begin{prop}[\cite{Kh}]\label{p2.1}
For any $n\geq 1$ and $(a_1,\cdots, a_n)\in \mathbb{N}^n$,
\begin{equation}\label{length-int}
  \left(2^n \prod_{k=1}^na_k\right)^{-2} \le |I_n(a_1,\cdots,a_n)|\le \left(\prod_{k=1}^na_k\right)^{-2}.
\end{equation}

% let $q_n$
%be given recursively by (\ref{ff2.1}).
% The cylinder $I_n(a_1,\cdots,a_n)$ is an interval with the endpoints $p_n/q_n$ and $(p_n+p_{n-1})/(q_n+q_{n-1})$.
%Then%\begin{equation}\label{ff10}
%%I_n(a_1,a_2,\cdots,a_n)= \left\{
%%\begin{array}{ll}
%%         \left[\frac{p_n}{q_n}, \frac{p_n+p_{n-1}}{q_n+q_{n-1}}\right),    & {\rm when }\ \
%%         n\ {\rm{is\ even}},\\
%%         \left(\frac{p_n+p_{n-1}}{q_n+q_{n-1}}, \frac{p_n}{q_n}\right],    & {\rm when }\ \
%%         n\ {\rm{is\ odd}}.
%%\end{array}
%%        \right.
%%\end{equation}
%\begin{equation}\label{5}
%\frac{1}{2q_n^2}\le \Big|I_n(a_1,\cdots,
%a_n)\Big|=\frac{1}{q_n(q_n+q_{n-1})}\le
%\frac{1}{q_n^2}.\end{equation}
%For each $n\ge 1$, $q_n(a_1, \cdots,
%a_n)\ge 2^{(n-1)/2}$ and \begin{equation}\label{ff13}
%\prod_{k=1}^na_k\le q_n(a_1,\cdots,a_n)\le   2^n \prod_{k=1}^na_k.
%\end{equation}
\end{prop}

The following lemma is used to calculate the lower bound of the Hausdorff dimension of $\overline{E}(\psi)$. 

Let $\{s_n\}_{n\geq 1}$ be a sequence of
integers and $\ell\geq 2$ be some fixed integer. Set$$
F(\{s_n\}_{n=1}^{\infty};\ell):=\big\{x\in [0,1): s_n\leq a_n(x)<\ell
s_n, \ {\rm{for \ all}}\ n\geq 1\big\}.
$$\begin{lem}[\cite{FLWW}]\label{lemma-FLWW}
Under the assumption that $ s_n\to \infty$ as
$n\to \infty$, one has\begin{eqnarray*}
\dim_HF(\{s_n\}_{n=1}^{\infty};\ell)=\left(2+\limsup_{n\to \infty}{\log s_{n+1} \over \log s_1s_2\cdots s_n}\right)^{-1}.
\end{eqnarray*}
\end{lem}


In fact, Lemma \ref{lemma-FLWW} has a more general form.  Let $s:=\{s_n\}_{n\geq 1}$ and $t:=\{t_n\}_{n\geq 1}$ be two sequences of
real numbers such that $s_n> 1, t_n >1$ for all $n\geq 1$.
Consider the following set
\[
F(s,t):=\big\{x\in [0,1): s_n\leq a_n(x)<
s_nt_n, \ {\rm{for \ all}}\ n\geq 1\big\}.
\]
\begin{lem}\label{lemma-general}
Assume that $ s_n\to \infty$ as
$n\to \infty$, and 
\[
\lim_{n\to\infty} {\log (t_n-1) \over \log s_n} =0.
\]
Then
\begin{eqnarray*}
\dim_HF(s,t)=\left(2+\limsup_{n\to \infty}{\log s_{n+1} \over \log s_1s_2\cdots s_n}\right)^{-1}.
\end{eqnarray*}
\end{lem}
The proof of Lemma \ref{lemma-general} is essentially contained in the proof of the lower bound of the dimension of $\underline{E}(\psi)$ in Subsection \ref{Dim-E-under}. So the details are left for the reader. A special case of Lemma \ref{lemma-general} can be found in \cite{LR}.

%\begin{lem}[\cite{FLWW}]\label{l2.7}
%$$\dim_H\left\{x\in [0,1): \limsup_{n\to \infty}\frac{\log q_n(x)}{n}=\infty\right\}= \frac{1}{2}.$$
%\end{lem}
The next lemma is useful for the upper bound of the Hausdorff dimensions of $\overline{E}(\psi)$ and $\underline{E}(\psi)$.
\begin{lem}[\cite{Lu}\label{Luczak}] For any $a>1, b>1$,$$
\dim_H\{x:a_n(x)\geq a^{b^n}, \forall n\geq 1\}=\dim_H\{x:a_n(x)\geq
a^{b^n}, {\text{i.o.}}\}=\frac{1}{b+1}.
$$ \end{lem}


\section{Proofs}
\subsection{Dimension of $\overline{E}(\psi)$}

We first calculate the Hausdorff dimension of $\overline{E}(\psi)$.
%\begin{proof}[Dimension of $\overline{E}(\psi)$]
 Recall that
$$ \overline{E}(\psi)=\Big\{x\in
[0,1):\limsup_{n\to\infty}\frac{\log a_1(x)+\cdots+\log
a_n(x)}{\psi(n)}=1\Big\}.
$$
We will only give the proof for $1<b< \infty$. The case $b=\infty$ can be obtained by a standard limit procedure.

Upper bound: For $x\in \overline{E}(\psi)$,
 let $S_n(x):= \log a_1(x) + \cdots + \log a_n(x) $.
  Then for any $\delta>0$, there are infinitely many $n$'s such that $S_n(x) \geq \psi(n)(1-\delta)$. This implies that there exist infinitely many $i\leq n$ such that $$\log a_i(x) \geq {\psi(n)\over n}(1-\delta).$$
  By the definition of $b$, for any $\epsilon>0$, $\psi(n)>(b-\epsilon)^n$ for all $n\geq 1$. Thus, we have infinitely many $i$'s, such that
  \[\log a_i(x)> {(b-\epsilon)^n\over n}(1-\delta)>(b-2{\epsilon})^i. \]
By Lemma \ref{Luczak},
% Then for any $\epsilon>0$, there are infinitely many $i$ such that
% \[ a_i \geq e^{(b-\epsilon)^i}. \]
 the Hausdorff dimension of $\underline{E}(\psi)$ is bounded by $1/(1+(b-2\epsilon))$ from above. Letting $\epsilon \to 0$, we obtain the upper bound.
 
 \medskip
 Lower bound:
%Assume that $\psi(1)$ is large so that $\psi(1)^{b-1}\geq 2$. 
We define a real sequence
$\{\tilde{c}_n\}_{n=1}^{\infty}$ as follows. Let $\tilde{c}_1=e^{\psi(1)}$ and $$
\tilde{c}_2=\min\left\{\frac{e^{\psi(2)}}{\tilde{c}_1}, \ \tilde{c}_1^{b-1+\epsilon}\right\}.
$$Assume that $\tilde{c}_n$ has already been well defined, then set $$
\tilde{c}_{n+1}=\min\left\{\frac{e^{\psi(n+1)}}{\prod_{k=1}^n\tilde{c}_k}, \
\prod_{k=1}^n\tilde{c}_k^{b-1+\epsilon}\right\}.
$$
Now for all $n\geq 1$, take $c_n=\lfloor\tilde{c}_n\rfloor+2$, where $\lfloor\cdot \rfloor$ stands for the integer part.
Then we can check that 
\begin{equation}\label{dim-cal}
\limsup_{n\to\infty}\frac{\log c_{n+1}}{\log c_1+\cdots+\log
c_n}\leq b-1+\epsilon.
\end{equation}
By the definition of $b$, we can further check that there exist infinitely many $n$, such that 
$\tilde{c}_{n+1}=\frac{e^{\psi(n+1)}}{\prod_{k=1}^n\tilde{c}_k}$. Thus we have
\begin{equation}\label{subset-Ec}
\limsup_{n\to\infty}\frac{\log c_1+\cdots+\log c_n}{\psi(n)}=1.
\end{equation}

Define
$$ E(\{c_n\}):=\{x\in [0,1): c_n \leq a_n(x)<2c_n , \ {\rm{for \ all}}
 \ n\geq 1\}.
$$
By (\ref{subset-Ec}), $E(\{c_n\}) \subset\overline{E}(\psi)$.


 To apply Lemma \ref{lemma-FLWW}, we need the condition $c_n\to \infty$ as $n\to \infty$ which is not necessarily satisfied. So, some modifications on the subset $E(\{c_n\})$ are needed. By the condition that $\psi(n)/n\to \infty$
as $n\to\infty$, we can choose a sequence $\{n_k\}_{k=1}^{\infty}$
such that for each $k\geq 1$, $$ \frac{\psi(n)}{n}\geq k^2, \
{\rm{when}}\ n\geq n_k.
$$
Take $\alpha_n=2$ if $1\leq n<n_1$ and
$$  \alpha_n=k+1, \ {\rm{when}} \ n_k\leq n<n_{k+1}.
$$Then it is easy to see 
\begin{eqnarray*}
\lim_{n\to\infty}\frac{\log \alpha_1+\cdots+\log
\alpha_n}{\psi(n)}=0 \ \ \text{and} \ \lim_{n\to\infty}\frac{\log
\alpha_{n+1}}{\log \alpha_1+\cdots+\log \alpha_n}=0.
\end{eqnarray*}
 Since $c_n\geq 2$ and $\alpha_n\geq 2$ for all $n\geq 1$,
we have $$\log c_n\leq \log (c_n+\alpha_n)\leq \log c_n+\log
\alpha_n \quad \forall n\geq 1.$$ So, by taking $s_n=c_n+\alpha_n$ for each $n\geq 1$,
we get
$$
\limsup_{n\to\infty}\frac{\log s_1+\cdots+\log
s_n}{\psi(n)}=1.
$$
Define $$ E(\{s_n\}):=\{x\in [0,1):s_n\leq a_n(x)<2s_n, \ {\rm{for \
all}}
 \ n\geq 1\}.
$$ Then $E(\{s_n\})\subset\overline{E}(\psi)$. As $s_n\to\infty$ as $n\to
\infty$, by Lemma \ref{lemma-FLWW}, we have $$ \dim_HE(\{s_n\})=\left(2+\limsup_{n\to \infty}{\log s_{n+1} \over \log s_1+\cdots+\log s_n}\right)^{-1}.
$$
Note that \begin{eqnarray*}&& \limsup_{n\to\infty}\frac{\log
s_{n+1}}{\log s_1+\cdots+\log
s_n}\\&=&\limsup_{n\to\infty}\frac{\log
(c_{n+1}+\alpha_{n+1})}{\log (c_1+\alpha_1)+\cdots+\log (c_n+\alpha_n)}\\
&\leq &\limsup_{n\to\infty}\frac{\log c_{n+1}+\log\alpha_{n+1}}{\log
(c_1+\alpha_1)+\cdots+\log (c_n+\alpha_n)}\\
&\leq &\limsup_{n\to\infty}\frac{\log c_{n+1}}{\log c_1+\cdots+\log
c_n}+\limsup_{n\to\infty}\frac{\log\alpha_{n+1}}{\log
\alpha_1+\cdots+\log \alpha_n}\\
&\leq &b-1+\epsilon.
\end{eqnarray*}Hence, $$
\dim_H\overline{E}(\psi)\geq \dim_HE(\{s_n\})\geq \frac{1}{b+1+\epsilon}.$$

%We have also proved that the set $\overline{E}(\psi)$ is always nonempty.
%\end{proof}
%$$\hfill $\Box$


\medskip
\subsection{Dimension of $\underline{E}(\psi)$}\label{Dim-E-under}
Recall that
$$ \underline{E}(\psi)=\Big\{x\in
[0,1):\liminf_{n\to\infty}\frac{\log a_1(x)+\cdots+\log
a_n(x)}{\psi(n)}=1\Big\}.
$$
As in the calculation of the Hausdorff dimension of $\overline{E}(\psi)$, we will only give the proof for $1<B< \infty$ and the easy case $B=\infty$ is left for the reader. 

% \begin{proof}[Proof of Theorem \ref{limsup}]
Upper bound: By the definition of $B$, for any $\epsilon>0$, there is a sequence $\{n_i\}$ such that 
\[
  \psi(n_i)> (B-\epsilon)^{n_i}.
\]
Denoting $S_n(x)= \log a_1(x) + \cdots + \log a_n(x)$, for all $x\in \underline{E}(\psi)$, for any $\delta>0$, we have 
$$S_n(x) \geq \psi(n)(1-\delta), \ \forall n\geq 1.$$
Thus 
\[S_{n_i}(x) \geq (B-\epsilon)^{n_i}(1-\delta).\]
Then there exists $j\leq n_i$ such that 
\[\log a_{j}(x) \geq (B-\epsilon)^{n_i}(1-\delta)/n_i > (B-2\epsilon)^{j}.\]
As $n_i$ goes to infinity, we will have infinitely many such $j$'s. Thus by Lemma \ref{Luczak}, the Hausdorff dimension of $\underline{E}(\psi)$ is bounded by $1/(1+(B-2\epsilon))$ from above.
 The upper bound then follows.
 
 \medskip
 Lower bound: We will construct a nonempty subset of $\underline{E}(\psi)$. Thus the following proof also shows that the set $\underline{E}(\psi)$ is always nonempty.
  
For any $\epsilon>0$, define
\[
A_i= \sup_{n\geq i}\exp\{\psi(n)(B+\epsilon)^{i-n}\}.
\]
This is the smallest function satisfying
\begin{equation}\label{prop-A}
A_{i+1} \leq A_i^{B+\epsilon} \quad \text{and} \quad A_i\geq e^{\psi(i)}.
\end{equation}
Let 
\[
Z:=\liminf\frac{\sum_{i=1}^n\log A_i}{\psi(n)}.
\]
Since for all $i\in \mathbb{N}$, $A_i\geq \exp\{\psi(i)(B+\epsilon)^{i-i}\} = e^{\psi(i)}$, we have
\[
Z\geq \frac{\log A_n}{\psi(n)}\geq 1.
\]
We start by showing the following proposition.
\begin{prop}\label{prop-finite}
We have $Z<\infty$.
\end{prop}
%\begin{proof}
Since $\limsup_{n\to\infty}\frac{\log \psi(n)}{n}=\log B$, we have for $n$ large enough,
\[
\psi(n)\leq (B+\epsilon/2)^n.
\]
Thus 
\[
A_i\leq \exp\{ (B+\epsilon/2)^n  (B+\epsilon)^{i-n}\}.%= e^{(B+\epsilon)^i}\exp\{ \left(\frac{B+\epsilon/2}{B+\epsilon}\right)^n\}
\]
Since $(B+\epsilon/2)^n/(B+\epsilon)^n$ goes to $0$ as $n\to\infty$, we have the supremum in the definition of $A_i$ can be obtained for the first time by some $t_i \geq i$.
Remark that for many consecutive $i$'s the $t_i$ will be the same. More precisely, $t_{i}=t_{i+1}=\cdots=t_{t_i}$.  Let us write $n_i=t_{t_i}$. Then $n_i<n_{i+1}$.  
 Notice that for these $n_i$, we have 
\[
\log A_{n_i}=\psi(n_i),
\]
and for $k\in (n_{i-1}, n_i]$,
\[
\log A_k= \psi(n_i)(B+\epsilon)^{k-n_i}.
\]
Thus 
\[
\sum_{k=n_{i-1}+1}^{n_i} \log A_k =\sum_{k=n_{i-1}+1}^{n_i}\psi(n_i)(B+\epsilon)^{k-n_i} \leq C \cdot \psi(n_i).
\]

Suppose $\{n_i\}$ are defined as above.
Denote $S_n\psi:= \sum_{k=1}^{n}\psi(k)$. Proposition \ref{prop-finite} follows directly from the following two lemmas.
\begin{lem}\label{lem-finite}
 The following liminf is finite:
\[
\liminf_{n\to\infty} \frac{S_n\psi}{\psi(n)}<\infty.
\]
\end{lem}
\begin{proof}
For $\epsilon>0$, we  will show that there exist infinitely many $i$, such that $\psi(i)> \epsilon S_{i-1}\psi$. If not, we will have
\[
S_n\psi = S_{n-1}\psi +\psi(n) \leq (1+\epsilon)S_{n-1}\psi.
\]
Thus 
\[
\limsup \frac{\log S_n\psi}{n} \leq \log (1+\epsilon),
\]
which is impossible since we have
\[
\limsup \frac{\log \psi(n)}{n}=B>\log (1+\epsilon).
\]
Write $l_i$ the sequence such that $\psi(l_i)> \epsilon S_{l_i-1}\psi$. Then 
\[
\frac{S_{l_i}\psi}{\psi(l_i)}=\frac{S_{l_i-1}\psi +\psi(l_i)}{\psi(l_i)} \leq 1+\frac{1}{\epsilon}<\infty,\]
and the conclusion follows.
\end{proof}


\begin{lem}\label{lem-subseq}
If 
\[
L:=\liminf_{n\to\infty} \frac{S_n\psi}{\psi(n)}<\infty,
\]
then 
\[
\liminf_{i\to\infty} \frac{S_{n_i}\psi}{\psi(n_i)}<\infty.
\]
\end{lem}
\begin{proof}
Let $m_k$ be the sequence such that 
\[
   \lim_{k\to\infty} \frac{S_{m_k}\psi}{\psi(m_k)} = L.
\]
Then each $m_k$ is in some $(n_{i-1}, n_i]$.
Thus 
\begin{align*}
S_{n_i}\psi =S_{m_k}\psi+ \sum_{j=m_k+1}^{n_i}\psi(j)&\leq (L+\epsilon) \psi(m_i) +\sum_{j=m_k+1}^{n_i}\psi(j)\\
&\leq (L+\epsilon) \sum_{j=m_k}^{n_i}\psi(j).
\end{align*}
Since for $j\in [m_k, n_i] \subset (n_{i-1}, n_i]$,
\[
\psi(j) \leq \frac{1}{(B+\epsilon)^{n_i-j}}\psi(n_i),
\]
we have
\[
S_{n_i}\psi  \leq C \cdot (L+\epsilon) \psi(n_i).
\]
Then the result follows.
\end{proof}
% Lemma \ref{lem-subseq}. %\qed
%\end{proof}

\medskip
We continue the estimation of the lower bound.
%Thus \[A_i \leq \exp\{(B+\epsilon)^{i}\}.\]
%
%Let $K=\exp\{\frac{1}{1-1/(B+\epsilon)}\}$.
Let $\epsilon_i$ be a sequence decreasing to $0$. (We will see $\epsilon_i=1/i$ are OK.) Construct $x$ by choosing $a_i(x)$ in the interval
\[
[A_i^{1/Z}(1-\epsilon_i), \ A_i^{1/Z}(1+\epsilon_i)].
\]
Choose $\epsilon_i$ such that 
\[ \lim_{n\to\infty}\frac{\sum_{j=1}^n \log (1\pm \epsilon_j)}{\psi(n)}=0.\]
Then
\[
   \liminf_{n\to\infty} \frac{\sum_{j=1}^n\log a_j(x)}{\psi(n)}= \liminf_{n\to\infty} \frac{\frac{1}{Z}\sum_{j=1}^n\log A_j}{\psi(n)}=1. 
\]
So such constructed $x$'s are indeed in the set $\underline{E}(\psi)$. Denote by $E$ the set of those $x$'s.

To estimate the Hausdorff dimension, we define a probability measure $\mu$ on $E$. For each position, we distribute the probability evenly.
That is for each possible $a_i$, we give the probability
\[ \frac{1}{|[A_i^{1/Z}(1-\epsilon_i), A_i^{1/Z}(1+\epsilon_i)]|}=\frac{1}{2\epsilon_i A_i^{1/Z}}.\]

Thus for each basic interval $I_n=I_n(a_1,\dots a_n)$, we have
\[
\mu(I_n)=\prod (2\epsilon_i A_i^{1/Z})^{-1}.
\]
By (\ref{length-int}) 
\[
|I_n| \approx \prod (A_i^{1/Z})^{-2}.
\]
To calculate the local dimension of $x\in E$, we will use a smaller interval $D_n$ included in $I_n$:
\[
  D_n=\cup_{a_{n+1}\geq A_{n+1}^{1/Z}(1-\epsilon_{n+1})} I_{n+1}(a_1, \cdots, a_n a_{n+1}).
\]
Since $a_i\in [A_i^{1/Z}(1-\epsilon_i), \ A_i^{1/Z}(1+\epsilon_i)]$, and $A_i$ grows super-exponentially, the Hausdorff dimension will be determined by calculating the local dimension 
\[
\liminf \frac{\log \mu(D_n)}{\log |D_n|}.
\]
(See Section 4 of Jordan and Rams \cite{JR}.)

 The length of this interval is
\[
|D_n|\approx |I_n|\cdot A_{n+1}^{-1/Z}.
\]
We have 
\[
-\log \mu(D_n)=-\log \mu(I_n) =\sum_{i=1}^n \log (2\epsilon_i) + \frac{1}{Z}\sum_{i=1}^n \log A_i. 
\]
Let us choose $\epsilon_i$ such that $|\sum_{i=1}^n \log (2\epsilon_i) |\ll \log A_{n+1}$. By the property that $A_{i+1}\leq A_i^{B+\epsilon}$, we deduce that
for big $n$,
\[
-\log \mu(D_n) \geq \sum_{i=1}^n \log (2\epsilon_i)+ \frac{1}{Z} \sum_{i=1}^n \frac{1}{(B+\epsilon)^i} \log A_{n+1} \approx \frac{1}{Z(B+\epsilon-1)} \log A_{n+1}.
\]
Now we calculate $-\log |D_n|$:
\[
-\log |D_n|\approx -\log |I_n| + \frac{1}{Z}\log A_{n+1} \approx -2 \log \mu(D_n) +\frac{1}{Z}\log A_{n+1}.
\]
Thus
\begin{eqnarray*}
\frac{-\log \mu(D_n)}{-\log |D_n|} &\approx& \frac{-\log \mu(D_n)}{  -2 \log \mu(D_n) +\frac{1}{Z}\log A_{n+1}} = \frac{1}{2+\frac{\frac{1}{Z} \log A_{n+1}}{-\log \mu(D_n)}}\\
& \geq& \frac{1}{2+B+\epsilon -1}=\frac{1}{B+1+\epsilon}.
\end{eqnarray*}
Then the lower bound follows from the Frostman Lemma (see \cite{Fa1}).
 %\end{proof}

\bigskip
\noindent{\bf Acknowledgements.} 
The authors thank Bao-Wei Wang for the fruitful discussions.
L. Liao was partially supported by the ANR, grant 12R03191A -MUTADIS (France).
M.Rams was partially supported by the MNiSW grant N201 607640 (Poland).

 {\small\begin{thebibliography}{99}


%\bibitem{Bi} P. Billingsley, Ergodic Theory and Information, John
%Wiley, New York, 1965.
%
%\bibitem{BiH} P. Billingsley and I. Henningsen, {\it Hausdorff dimension of some continued-fraction sets}, Z.
%Wahrscheinlichkeitstheorie verw. Geb. 31 (1975) 163-173.
%

\bibitem{Fa1} K. J. Falconer, Fractal Geometry, Mathematical Foundations
and Application, Wiley, 1990.

%\bibitem{FLM} A. H. Fan, L. M. Liao, and J. H. Ma, {\it On the frequency of partial quotients of regular continued fractions}, Math. Proc. Camb. Phil. Soc., 148 (2010), 179-192,
%
\bibitem{FLWW} A. H. Fan, L. Liao, B. W. Wang and J. Wu, {\it On Kintchine exponents and Lyapunov exponents
of continued fractions}, Ergod. Th. Dynam. Sys., 29 (2009), 73-109.

\bibitem{FLWW13} A. H. Fan, L. Liao, B. W. Wang and J. Wu, {\it On the fast Khintchine spectrum in continued fractions}, Monatshefte f\"ur Mathematik, 171(2013), 329-340.


%\bibitem{FWLT} D. J. Feng, J. Wu, J. C. Liang and S. Tseng, Appendix to the paper by T. L\'uczak---a simple proof
%of the lower bound: "On the fractional dimension of sets of
%continued fractions", Mathematika 44 (1) (1997), 54-55.

%\bibitem{Go} I. J. Good, {\it The fractional dimensional theory of continued fractions},
%Proc. Camb. Philos. Soc., 37 (1941), 199-228.
%
%\bibitem{HW} G. Hardy and E. Wright, An Introduction to the Theory of
%Numbers, fifth edition, Oxford University Press, 1979.
%
%
%\bibitem{Hi} K. E. Hirst, {\it A problem in the fractional dimension theory of continued fractions.} Quart. J. Math. Oxford Ser. (2) 21 1970 29�35.
%
%
%\bibitem{Ja} I. Jarnik, {\it Zur metrischen Theorie der diopahantischen
%Approximationen}, Proc. Mat. Fyz., 36 (1928), 91-106.
%
\bibitem{JR} T. Jordan and M. Rams, {\it Increasing digit subsystems of infinite iterated function systems}. Proc. Amer. Math. Soc. 140 (2012), no. 4, 1267-1279.
%
%\bibitem{IK} M. Iosifescu and C. Kraaikamp, Metrical Theory of Continued Fractions,
% Mathematics and its Applications, 547. Kluwer Academic Publishers, Dordrecht, 2002.
%
%\bibitem{KJ} M. Kesseb\"{o}hmer and J. Jaerisch, {\it The arithmetic-geometric scaling spectrum for continued fractions},
%Arkiv f\"{o}r Matematik 48 (2010), no. 2, 335-360.
%
%\bibitem{KS} M. Kesseb\"{o}hmer and O. Stratmann, {\it  A multifractal analysis for Stern-Brocot intervals, continued fractions and Diophantine growth rates},
%Journal f\"{u}r die reine und angewandte Mathematik, 605 (2007), 133-163.
%
%\bibitem{KZ} M. Kesseb\"{o}hmer and S. Zhu, {\it Dimension sets for infinite IFSs: Texan Conjecture},
%J. Number theory, 116 (2006), 230-246.
%

\bibitem{Kh} A. Ya. Khintchine, Continued Fractions, P. Noordhoff,
Groningen, The Netherlands, 1963.

\bibitem{LR} L. Liao and M. Rams, Subexponentially increasing sum of partial quotients in continued fraction expansions, preprint, arxiv.org/abs/1405.4747.

\bibitem{Lu}T. \L uczak, {\it On the fractional dimension of sets of continued fractions.} Mathematika 44 (1997), no. 1, 50�-53.


%\bibitem{Ma} D. Mayer, {\it On the thermodynamics formalism for the Gauss
%map}, Comm. Math. Phys., 130 (1990), 311-333.


%\bibitem{PW} M. Pollicott and H. Weiss, {\it Multifractal analysis of Lyapunov exponent for
%continued fraction and Manneville-Pomeau transformations and
%applications to Diophantine approximation}, Comm. Math. Phys., 207
%(1) (1999), 145-171.
%

%\bibitem{Wa} P. Walter, An Introduction to Ergodic Theory. Graduate Texts in Mathematics, 79. Springer-Verlag, New York-Berlin, 1982.

%\bibitem{WW} B.W. Wang and J. Wu, {\it Hausdorff dimension of certain sets arising in continued fraction
%expansions}, Adv. Math., (2008).

%\bibitem{Wu} J. Wu, A remark on the growth of the denominators of
%convergents, Monatsh. Math., 147 (3) (2006), 259-264.
\end{thebibliography}

\end{document}
