
We have argued that $\mu$-functors are closed under parameterized
initial algebras: by the Beki\v{c} property, it becomes possible to
construct initial solutions of systems of functorial equations by
means of $\mu$-terms. The arising algebraic expressions representing
the solution of a system are however large in the dimension of the
system and are not unique.  Moreover, a large algebraic expression
could be useless for understanding its denotation in concrete
categories.  For this reason we would like to have some kind of
``smooth'' terms for the theory of $\mu$-bicomplete categories. These
terms should have a compact representation and possibly they should be
suggestive of their semantics.  To achieve this goal, the central
notion is that of parity game, cf.  for example \cite{zie,AN01}. We
define it here in a slightly generalized way.
\begin{dfntn}
  A \emph{parity game} is a tuple $G = \langle S,\hg,\kappa
  ,\epsilon\rangle$, where
  \begin{itemize}
  \item $S =\langle \dom,\cod : M \rTo P \rangle$ is a finite graph of
    positions and moves.
  \item $\hg: P \rTo \agset{n}$ is a function such that, if $\hg(p)
    = \omega$, then $\{ \,m\,| \,\dom(m) = p \,\} = \emptyset$.  
  \item $\kappa : \set{n} \rTo \{ \mu, \nu \}$.
  \item $\epsilon: \{\, p \in P \,| \,\hg(p) \neq \omega \,\} \rTo \{
    \sigma,\pi \}$.
  \end{itemize}
\end{dfntn}

We fix some terminology and notation. If $\hg : P \rTo \agset{n}$,
then we shall say that $n$ is the \emph{height} of $G$ and write
$\height{G} = n$.  For each $p \in P$, we let $M_{p}$ be the set
$\dom^{-1}(p)$. We let $P_{i} = \hg^{-1}(i)$, $P_{<i} = \bigcup_{j <
  i} P_{j}$, $P_{\leq i} = \bigcup_{j \leq i} P_{j}$ for $i \in
\agset{n}$. We shall also use the notation $P_{\geq i}$ for the set
$\bigcup_{j \geq i} P_{j}$. Unless specified we will assume that the
underlying structure of a given parity game $G$ is the tuple $\langle
S,\hg,\kappa ,\epsilon\rangle$, $S$ being the graph $\langle
P,M,\dom,\cod\rangle$. A pointed parity game is a pair $\langle G, p
\rangle$ where $G$ is a parity game and $p \in P$.

\vspace{0.5em}

We interpret the above data as a two person game $G(E)$, parameterized
in a choice of sets $E = \{ E_{x} \}_{x \in P_{\omega}}$.  The graph
$S$ is a board with a set of positions $P$ and a set of allowed moves
$M$. A move $m \in M$ is from position $\dom(m)$ to position
$\cod(m)$; observe that we allow different moves relating the same
pair of positions; also, the two players need not to alternate.  From
a position $p$ the set of moves $M_{p}$ is available and player
$\epsilon(p)$ among players $\sigma$ and $\pi$ must choose how to
move.  The normal play condition holds: if a player cannot move, then
he loses.  On an infinite play $\gamma = \gamma_{0} \rightarrow
\gamma_{1}\rightarrow \ldots \gamma_{n} \rightarrow \ldots $ we will
be able to find regions among $P_{1},\ldots ,P_{n}$ which are visited
infinitely often, and among them we will be able to pick a region
$P_{i}$ with $i$ maximal.  Then, this infinite path is a win for
player $\sigma$ if and only if $i$ is colored by $\nu$. More formally,
if we let
\begin{eqnarray}
  \label{eq:in}
  \In \gamma & = & \{\, i \in \set{n} \,|\,\card \{\, l\,|\,\hg(
  \gamma_{l} ) = i \,\} = \omega \,\}\,, 
\end{eqnarray}
then $\gamma$ is a  win for player $\sigma$ if and only if
\begin{eqnarray*}
\kappa(\, \max \In \gamma\,) & = & \nu\,.
\end{eqnarray*}
If a play ends in a position $x \in P_{\omega}$, then player $\sigma$
must choose an element $e \in E_{x}$, and then he wins. If $E_{x} =
\emptyset$, then he loses.

We remark that if $P_{\omega} = \emptyset$, then the above data and
game theoretic interpretation coincide with the usual one \cite[\S
4]{AN01}.  From a game theoretic point of view, we are allowed to
normalize the functions $\hg$ and $\kappa$, so that we can always
suppose that $\kappa(i) = \nu$ if and only if $i$ is odd. In this case
an infinite play $\gamma$ is a win for player $\sigma$ if and only if
the region visited infinitely often in $\gamma$ of maximal height is
odd, whence the name of ``parity game''.

In the theory of automata recognizing infinite objects the way of
specifying a set of infinite paths in a graph is called an acceptance
condition.  The acceptance condition by which we specify the set of
infinite winning plays for player $\sigma$ was introduced by Mostowski
in \cite{MR827531} and is also known as a Rabin chain condition. Under
the hypothesis that $\kappa(i) = \nu$ if and only if $i$ is odd and
that $\height{G} = 2n$, we can let $F_{k} = P_{\geq 2k-1}$ and $E_{k}
= P_{\geq 2k}$ for $k =1,\ldots ,n$. These sets form a decreasing
chain and moreover the pairs $(F_{k},E_{k})_{k = 1,\ldots ,n}$ are
Rabin pairs for the set of infinite winning plays for player $\sigma$,
meaning that an infinite path $\gamma$ is a win for player $\sigma$ if
and only if there exists $k \in \{1,\ldots ,n \}$ such that $\In
\gamma \cap F_{k} \neq \emptyset$ and $\In \gamma \cap E_{k} =
\emptyset$. There are other ways of characterizing this acceptance
condition by means of Muller tables \cite{MR95a:90238}, but we won't
investigate this subject further. On the other hand we recall that
parity games are essential tools for model checking. For example, it
is shown in \cite{MR1826118} that the problem of deciding whether
player $\sigma$ has a winning strategy from a given position of a
parity game is equivalent -- under linear reduction -- to the problem
of deciding whether a $\mu$-calculus formula holds in a given model.

The goal of adding positions at infinite height is to make it possible
to analyze parity games inductively. The main tool for this is the
predecessor game of a parity game, whose construction we illustrate in
figure \ref{fig:P(G)}. The game on the right is obtained from the one
on the left by erasing all the moves from the region of maximal finite
height.
\begin{figure}[h]
  \centering
  $$
  \mygame[3em]{
    []!E{1}([r]!E{2}="A") [d]!O{3}([r]!P{4}) [d]!P{5}([r]!P{6})
    [d]!P{7}([r]!O{8}) "3"(:"1",:"5") "5"(:"7",:"6") "7"(:"6")
    "4"(:"3",:@/^0.2em/"6") "6"(:"8",:@/^0.7em/"4")
    "8"(:"7",:@/_2em/"2") "1"("2"!L{}{\omega}) "3"("4"!L{\nu}{3})
    "5"("6"!L{\mu}{2}) "7"("8"!L{\nu}{1}) }
  \hspace{4em}
  \mygame[3em]{ []!E{1}([r]!E{2}="A") [d]!E{3}([r]!E{4})
    [d]!P{5}([r]!P{6}) [d]!P{7}([r]!O{8}) 
    "5"(:"7",:"6") "7"(:"6") 
    "6"(:"8",:@/^0.7em/"4") "8"(:"7",:@/_2em/"2") "1"("4"!L{}{\omega})
    "5"("6"!L{\mu}{2}) "7"("8"!L{\nu}{1}) }
  $$
  \vspace{-11mm}
  \caption{On the left a parity game, on the right its 
    predecessor game.}
  \label{fig:P(G)}
\end{figure}

\begin{dfntn}
  If $G = \langle S,\hg,\kappa ,\epsilon\rangle$ is a parity game of
  height $n > 0$, then its \emph{predecessor game} $P(G)$, of height
  $n-1$, is obtained from $G$ by erasing all the moves from $P_{n}$.
  More precisely, $P(G) = \langle S',\hg',\kappa',\epsilon' \rangle$,
  where:
  \begin{itemize}
  \item $S' = \langle \dom,\cod: \dom^{-1}(P_{< n})
    \rTo P\rangle$.
  \item $\hg'(p) = \hg(p)$ if $\hg(p) < n$, otherwise $\hg'(p) =
    \omega$.
  \item For $i \in \set{n-1}$, we let $\kappa'(i) = \kappa(i)$.
  \item If $\hg'(p) < n$, then we let $\epsilon'(p) = \epsilon(p)$.
  \end{itemize}
\end{dfntn}

In the following we shall endow the data defining a parity game with
an algebraic meaning. We let $\Cat{C}$ be a fixed category with finite
products and finite coproducts.  If $G$ is a parity game, then for
each $p \in P_{< \omega}$ we let
$$
\begin{array}{cclcl}
  \prj(\cod,p)
  & = &  \ltuple \prj_{\cod(m)}
  \rtuple_{ m \in M_{p} }
  &:& \Cat{C}^{P} \rTo
  \Cat{C}^{M_{p}} \\[2mm]
  \EQ_{p} & = & 
  \left \{
    \begin{array}{@{\hspace{0.5em}}ll}
      \prod \circ \prj(\cod,p)\,,&\epsilon(p) =
      \pi\\[2mm]
      \coprod \circ \prj(\cod,p) \,,&\epsilon(p) = \sigma
    \end{array}\right.
  &:& \Cat{C}^{P} \rTo
  \Cat{C} \,.
\end{array}
$$
For $k = 1,\ldots ,\height{G}$ we let
$$
\begin{array}{cclcl}
  \EQ_{k} & = & 
  \ltuple \EQ_{p}\rtuple_{p \in P_{k}}
  &: & \Cat{C}^{P} \rTo
  \Cat{C}^{P_{k}}
\end{array}
$$
and finally we let $\EQ_{G} = \EQ_{\height{G}}$.



\begin{dfntn}
  We define a partial correspondence $\val{-}$,  mapping a parity
  game $G$ to a functor $\val{G}:\Cat{C}^{P_{\omega}} \rTo
  \Cat{C}^{P_{< \omega}}$, by induction on the height, as follows.  If
  $\height{G} = 0$, then $P_{< \omega} = \emptyset$ so that there is a
  unique choice of $\val{G}$.  Suppose that $\height{G} = n > 0$ and
  that $\val{P(G)}$ is defined. Let
    \begin{eqnarray*}
      F & = &\val{P(G)}\circ \prj_{\Cat{C}^{P_{n}}\times \Cat{C}^{P_{\omega}}},
    \end{eqnarray*}
    and consider the functor
    $$
    \Cat{C}^{P_{< n }} \times \Cat{C}^{P_{n}}\times
    \Cat{C}^{P_{\omega}} \rTo[l>=4em]^{\ltuple F,\, \EQ_{G} \rtuple }
    \Cat{C}^{P_{< n}} \times \Cat{C}^{P_{n}}\,.
    $$
    If $\kappa(n) = \mu$, then we let $\val{G}$ be the
    parameterized initial algebra of the above functor, otherwise, if
    $\kappa(n ) = \nu$, we let $\val{G}$ be its parameterized final
    coalgebra. If $\val{P(G)}$ is undefined or if the required initial
    algebras or final coalgebras do not exist, then $\val{G}$ is
    undefined. 
    We say that $\Cat{C}$ is \emph{complete with respect
      to parity games} if for each parity game $G$, the functor
    $\val{G}:\Cat{C}^{P_{\omega}} \rTo \Cat{C}^{P_{< \omega}}$ is
    defined. 
\end{dfntn}

Whenever the functor $\val{G}: \Cat{C}^{P_{\omega}} \rTo
\Cat{C}^{P_{<\omega}}$ is defined, it is useful to extend it to a
functor $\sval{G}:\Cat{C}^{P_{\omega}} \rTo \Cat{C}^{P}$, in the
obvious way, by setting
$$
\sval{G} = \langle \val{G},\id_{\Cat{C}^{P_{\omega}}}\rangle
: \Cat{C}^{P_{\omega}} \rTo \Cat{C}^{P_{<\omega}}\times \Cat{C}^{P_{\omega}}\,.
$$
Observe that, according to proposition \ref{prop:bekic3}, the
functor $\prj_{\height{G}} \circ \val{G}$ is a parameterized initial
algebra (or final coalgebra) of the functor $\EQ_{G}\circ
\sval{P(G)}$. Moreover, according to the same proposition, the value
of $\val{G}$ is completely determined up to natural isomorphism by
$\prj_{\height{G}} \circ \val{G}$ and $\val{P(G)}$. Hence, in order to
prove that $\val{G}$ and $\val{H}$ are naturally isomorphic, it is
enough to prove that $\EQ_{G}$ and $\EQ_{H}$ are naturally isomorphic,
that $\kappa(\height{G}) = \kappa(\height{H})$, and that $\val{P(G)}$
is naturally isomorphic to $\val{P(H)}$.

\begin{dfntn}
  We say that a functor $F: \Cat{C}^{I} \rTo \Cat{C}^{J}$ is a
  \emph{parity functor} if it is naturally isomorphic to a functor of
  the form $\prj_{J} \circ \sval{G}$, where $G$ is a parity game such
  that $P_{\omega} = I$ and $J \subseteq P$ is a subset of positions.
\end{dfntn}
If in the previous lemma $I = \{p \}$ is a singleton, we will use the
notation $\sval{G}_{p}$ for the functor $\prj_{p} \circ \sval{G}$.

In the following two lemmas, needed in the proof of proposition
\ref{prop:termstogames}, we exemplify how game theoretical ideas lift
to the algebra.  In \ref{def:suspension} we introduce two
constructions which respectively introduce and eliminate holes in the
height. The first construction is exemplified in figure
\ref{fig:suspension}. In \ref{def:normalized} and
\ref{lemma:normalized} we show that regions of contiguous heights can
always be assumed to be non empty and have different colors $\{\mu,\nu
\}$.  These constructions are shown to be algebraic invariants.
\begin{figure}[h]
  \centering
  $$
  \mygame[3em]{
    []!E{1}([r]!E{2}="A") [d]!O{3}([r]!P{4}) [d]!P{5}([r]!P{6})
    [d]!P{7}([r]!O{8}) "3"(:"1",:"5") "5"(:"7",:"6") "7"(:"6")
    "4"(:"3",:@/^0.2em/"6") "6"(:"8",:@/^0.7em/"4")
    "8"(:"7",:@/_2em/"2") "1"("2"!L{}{\omega}) "3"("4"!L{\nu}{3})
    "5"("6"!L{\mu}{2}) "7"("8"!L{\nu}{1}) }
  \hspace{4em}
  \mygame[3em]{
    []!E{1}([r]!E{2}="A") 
    [d]!O{3}([r]!P{4}) [d(2)]!P{5}([r]!P{6})
    [d]!P{7}([r]!O{8})
    "3"[d]!I{9}([r]!I{10})
    "3"(:"1",:"5") "5"(:"7",:"6") "7"(:"6")
    "4"(:"3",:@/^0.2em/"6") "6"(:"8",:@/^0.7em/"4")
    "8"(:"7",:@/_2em/"2") "1"("2"!L{}{\omega}) "3"("4"!L{\nu}{4})
    "5"("6"!L{\mu}{2}) "7"("8"!L{\nu}{1})
    "9"("10"!L{\mu}{3})
  }
  $$
  \vspace{-11mm}
  \caption{A game $G$ on the left and 
    the game $G_{3,\mu}$ on the right.}
  \label{fig:suspension}
\end{figure}

In the following definition, let $\hat{\imath} : \{1,\ldots ,n\} \rTo
\{1,\ldots ,n + 1 \}$ be the unique order preserving injection which
avoids $i \in \{1,\ldots ,n + 1\}$.
\vfill\eject\noindent
\begin{dfntn}
  \label{def:suspension}
  Let $G = \langle S,\hg,\kappa,\epsilon \rangle$ and suppose  that
  $\height{G} = n$. 
    \begin{itemize}
  \item For $i = 1,\ldots,n + 1$ and $\theta \in \{\mu,\nu \}$, we
    define $G_{i,\theta} = \langle
    S,\hg_{i},\kappa_{i,\theta},\epsilon \rangle$ by letting
    $\hg_{i} = \hg \comp \hat{\imath}$ and , $\kappa_{i,\theta}(j) =
    \kappa(j)$ if $j < i$, $\kappa_{i,\theta}(i) = \theta$ and
    $\kappa_{i,\theta}(j) = \kappa(j-1)$ if $j > i$.
  \item We define $G_{\bullet} = \langle
    S,\hg_{\bullet},\kappa_{\bullet},\epsilon \rangle$ as follows: we
    let $\hg_{\bullet} \comp j$ be the unique factorization of $\hg$
    such that $\hg_{\bullet}: P \setminus P_{\omega} \rOnto \{1,\ldots
    ,k \}$ is surjective and $j :\{1,\ldots ,k \} \rInto \{1 ,\ldots
    ,n \}$ is injective and order preserving; we let
    $\kappa_{\bullet} = j \comp \kappa$.
  \end{itemize}
\end{dfntn}
Observe that $\height{G_{i,\theta}} = n + 1$ and that, in the game
$G_{\bullet}$, $P_{j} \neq \emptyset$ for $j = 1,\ldots ,
\height{G_{\bullet}}$.
\begin{lmm}
  \label{lemma:suspensions}
  There exist natural isomorphisms $\val{G} \iso
  \val{G_{i,\theta}}$ and $\val{G} \iso \val{G_{\bullet}}$.
\end{lmm}
\begin{proof}
  The isomorphism $\val{G_{n + 1,\mu}} \iso
  \val{G}$
  follows by observing that $P(G_{n+1,\mu}) 
  \linebreak 
  = G$ and by letting
  $\Cat{C}$ be $\Cat{C}^{P_{\leq n}}$, $\Cat{D} = \Cat{C}^{P_{n + 1}}
  = \Cat{C}^{\emptyset} = 1$, $\Cat{E} = \Cat{C}^{P_{\omega}}$ in
  proposition \ref{prop:bekic2}: the left projection of $\val{G_{n +
      1,\mu}}: \Cat{C}^{P_{\omega}} \rTo \Cat{C}^{P_{\leq n}} \times
  1$ is computed as $\val{G}$. An analogous observation shows that
  there is an isomorphism $\val{G_{n + 1,\nu}} \iso \val{G}$.
  
  If $i \leq n$, then we can reason by induction on the height,
  observing that $P(G_{i,\theta}) = P(G)_{i,\theta}$ and
  $\EQ_{G_{i,\theta}} = \EQ_{G}$, so that $\EQ_{G_{i,\theta}} \circ
  \sval{P(G_{i,\theta})}$ and $\EQ_{G} \circ \sval{P(G)}$ are
  naturally isomorphic.
  
  On the other hand, we argue that $\val{G}$ and $\val{G_{\bullet}}$
  are naturally isomorphic as follows: $j$ can be factored by a
  sequence of the functions $\hat{\imath}$, hence $G$ can be obtained
  from $G_{\bullet}$ by a sequence of the operations
  $(-)_{i,\theta}$ and the result follows from our previous
  considerations.  
\end{proof}




\begin{dfntn}
  \label{def:normalized}
  We say that a parity game $G$ is \emph{normalized} if $\kappa(i)
  \neq \kappa(i + 1)$ for $i = 1,\ldots ,\height{G}-1$ and $P_{i} \neq
  \emptyset$ for $i = 1,\ldots ,\height{G}$.
\end{dfntn}
\begin{lmm}
  \label{lemma:normalized}
  For each parity game $G$ there exists a normalized parity game
  $N(G)$ on the same set of positions of $G$ such that $\val{G} \iso
  \val{N(G)}$.
\end{lmm}
\begin{proof}  
  Let $G = \langle S,\hg,\kappa,\epsilon \rangle$ be a game with
  $\height{G} = n + 1 > 1$. We first define a game $G_{N}$.  If
  $\kappa(n + 1) = \kappa(n)$, then we let $G_{N} = \langle
  S,\hg_{N},\kappa,\epsilon \rangle$, where $\hg_{N}(p) = n$ if
  $\hg(p) = n + 1$ and otherwise $\hg_{N}(p) = \hg(p)$. To verify that
  $\val{G}$ is isomorphic to $\val{G_{N}}$, observe that $P(G_{N}) =
  P(P(G))$ and that $\EQ_{G_{N}} = \ltuple \EQ_{P(G)},\EQ_{G} \rtuple$
  and therefore let $F = \ltuple \val{P(P(G))} \circ \prj_{P_{\geq
      n}}, \EQ_{P(G)}\rtuple$ and $G = \EQ_{G}$ in the statement of
  the Beki\v{c} property \ref{prop:bekic}.  Otherwise, if $\kappa(n +
  1) \neq \kappa(n)$, then we let $G_{N} = G$.
  
  We define then $N(G)$ by induction on the height.  If $\height{G}
  \leq 1$, then $N(G) = G$. Otherwise, in order to obtain $N(G)$, we
  first construct $N(P(G))$, and then a game $G'$ by adding to the
  region of maximal height of $G$ transitions so that $\EQ_{G'} =
  \EQ_{G}$. As the last step, we let $N(G) = G'_{N}$. By induction, it
  is shown that $\val{N(G)} = \val{G}$.  
\end{proof}



The following theorem is the main result of this section and
generalizes to categories the well known fact that a vectorial
$\mu$-calculus has no more expressive power of its scalar version
\cite[\S 2.7]{AN01}.
\begin{thrm}
  \label{prop:iff}
  A category is $\mu$-bicomplete with respect to parity games if and
  only if it is $\mu$-bicomplete.
\end{thrm}
In order to prove the theorem we translate parity functors into
collections of $\mu$-terms and vice-versa we represent $\mu$-terms
by pointed parity games.  To show that this translation is sound the
main tool is the Beki\v{c} property discussed in section
\ref{sec:bekic}.

\begin{prpstn}
  \label{prop:gamestoterms}
  For each parity game $G$ we can find a collection of $\mu$-terms $\{
  s_{p} \}_{p \in P}$, such that $s_{p} \in
  \mu\mathcal{T}(P_{\omega})$ and
  \begin{eqnarray*}
    \sval{G} & := & \ltuple \,\val{s_{p}} \,\rtuple_{p \in P}\,.
  \end{eqnarray*}
\end{prpstn}
The meaning of the symbol $:=$ is that the functorial expression on
the right determines the existence of the functorial expression on the
left. That is, natural transformations (needed as projections,
injections and as the structure part of initial algebras or final
coalgebras) can be constructed out of the natural transformations
given with the interpretations of the $\mu$-terms, so that the
functorial expression on the right together with these new natural
transformations have the universal property that determines the
left-hand side of the equation up to canonical isomorphism. Thus it
follows:
\begin{crllr}
  If $\Cat{C}$ is a $\mu$-bicomplete category, then
  $\Cat{C}$ is  complete w.r.t. parity games.
\end{crllr}
\begin{proof}[Proof of proposition \ref{prop:gamestoterms}]
  Clearly, it is enough to find a collection of $\mu$-terms indexed by
  $P_{< \omega}$ such that $\val{G} := \ltuple \,\val{s_{p}}
  \,\rtuple_{p \in P_{< \omega}}$, since then we can complete this
  collection to a collection representing $\sval{G}$, by letting
  $s_{p}$ be the $\mu$-term $p \in \mu\mathcal{T}(P_{\omega})$ if $p
  \in P_{\omega}$.

  If  $\height{G} = 0$, then the statement is true
  since $P_{ < \omega} = \emptyset$ and the empty collection of terms
  satisfies the requirements.
  
  Suppose that $\height{G} = n > 0$ and that that $\kappa(n) = \mu$.
  An analogous argument works if $\kappa(n) = \nu$.
  
  By the induction hypothesis there are $\mu$-terms $\{ s_{p} \}_{p
    \in P}$ with $s_{p} \in \mu\mathcal{T}(P_{n} \cup
  P_{\omega})$ such that $\sval{P(G)} = \ltuple \,\val{s_{p}}
  \,\rtuple_{p \in P}$.
  According to proposition \ref{prop:bekic2}, we can construct the
  functor $\val{G}$ by means of $\mu$-terms, provided we are able to
  show that the functor $\EQ_{G} \circ \sval{P(G)}$
  admits a parameterized initial algebra which is representable by
  means of $\mu$-terms. 
  We prove this by induction on the cardinality of $P_{n}$.
  If $P_{n} = \emptyset$, then there is nothing to prove.
  Otherwise, pick $p_{0} \in P$, let $P'_{n} = P_{n} \setminus \{
  p_{0} \}$ and represent the functor $\EQ_{G}$ as $\langle \EQ_{P'_{n}}, \EQ_{p_{0}} \rangle$
  where $\EQ_{P'_{n}} = \ltuple \, \EQ_{p} \, \rtuple_{p \in P'_{n}}$.
  We claim that an initial algebra of the functor $\EQ_{P'_{n}} \circ
  \sval{P(G)}$ exists and is constructible by means of $\mu$-terms.
  Indeed a parity game $G'= \langle S' ,\hg',\kappa'
  ,\epsilon'\rangle$ on the same set of positions and with the same
  height as $G$, such that $\kappa'(i) = \kappa(i)$ for $i =1,\ldots
  ,\height{G}$, $P(G') = P(G)$, $\hg'(p) = n $ if and only if $p \in
  P'_{n}$ and $\EQ_{G'} = \EQ_{P'_{n}}$, is easily constructed out of
  $G$.
  Since $\card P'_{n} < \card P_{n}$ by the induction hypothesis we
  have a desired representation of $\val{G'}$ by $\mu$-terms $t_{p}
  \in \mu\mathcal{T}(\{ p_{0} \} \cup P_{\omega})$, for $p \in P_{\leq
    n}\setminus \{p_{0}\}$. It follows that $\langle
  \val{t_{p}}\rangle_{p \in P_{n}'}$ is the desired representation of
  the initial algebra of $\EQ_{P'_{n}} \circ \sval{P(G)}$, since by
  proposition \ref{prop:bekic3} $\prj_{P'_{n}} \circ \sval{G'}$ is an
  initial algebra for $\EQ_{G'} \circ \sval{P(G')} = \EQ_{P'_{n}}
  \circ \sval{P(G)}$.

  Let
  $s : M_{p_{0}} \rTo \mu\mathcal{T}(P_{n} \cup P_{\omega})$ be
  the function defined by the relation
  \begin{eqnarray*}
    s(m) & = & s_{\cod m}
  \end{eqnarray*}
  and let $u \in \mu\mathcal{T}(P_{n} \cup
  P_{\omega})$ be the $\mu$-term defined as
  \begin{eqnarray*}
    u & = & 
    \left \{
      \begin{array}{cc}
        \Land[M_{p_{0}}] s\,, & \epsilon(p_{0}) = \pi \,\\
        \Lor[M_{p_{0}}] s\,, & \epsilon(p_{0}) = \sigma \,,
    \end{array}
    \right .
  \end{eqnarray*}
  then 
  \begin{eqnarray*}
    \EQ_{p_{0}} \circ \sval{P(G)}
    & = & \val{u} :
  \Cat{C}^{P_{n}}\times \Cat{C}^{P_{\omega}} \rTo \Cat{C}\,.
  \end{eqnarray*}

  We can now construct an initial algebra of $\EQ_{G} \circ
  \sval{P(G)}$ according to the Beki\v{c} property.  Let
  \begin{eqnarray*}
    v_{p_{0}} & = &  \mu_{p_{0}}.(\,u[
    t_{p}/p]_{p \in P'_{n}}\,)
  \end{eqnarray*}
  and for $p \in P'_{n}$ let
  \begin{eqnarray*}
    v_{p} & = & t_{p}[v_{p_{0}}/p_{0}]
  \end{eqnarray*}
  then the functor $\langle \val{v_{p}}\rangle_{p \in
    P_{n}}$  carries a canonical structure of an initial algebra for
  the functor $\EQ_{P_{n}} \circ \sval{P(G)}$.  
\end{proof}


\begin{prpstn}
  \label{prop:termstogames}
  For each $\mu$-term $s \in\mu\mathcal{T}(X)$ there exists a
  pointed parity game $\langle G,p\rangle$ such that $P_{\omega} = X$
  and 
  \begin{eqnarray*}
    \val{s} & := & \sval{G}_{p}\,.
  \end{eqnarray*}
\end{prpstn}
Again, the meaning of the symbol $:=$ is that the functorial
expression on the right can be endowed with a structure so that it has
the universal property which determines the left-hand side of the
equation up to canonical isomorphism. Thus it follows:
\begin{crllr}
  If $\Cat{C}$ is a category complete w.r.t. parity games, then
  $\Cat{C}$ is $\mu$-bicomplete.
\end{crllr}
\begin{lmm}
  Let $G$ be a parity game, we define $G^{p_{0}}$ to be the game
  obtained from $G$ by adding a new position $p_{0}$ to set of
  position at infinity $P_{\omega}$. Then there exists a natural
  isomorphism $\val{G^{p_{0}}} \iso \val{G}\circ
  \prj_{\Cat{C}^{P_{\omega}}}$.
\end{lmm}
\begin{proof}
  The observation is obvious if $\height{G} = 0$. On the other hand,
  if $\height{G} > 0$, then $P(G^{p_{0}}) = P(G)^{p_{0}}$ and
  $\val{P(G^{p_{0}})} \iso \val{P(G)} \circ \prj_{n,\omega}$, by
  induction. Moreover $\EQ_{G^{p_{0}}} = \EQ_{G} \circ
  \prj_{P}$, so that $\val{G^{p_{0}}}$ is the defined to be the
  initial algebra of the functor $\langle \val{P(G)} \circ
  \prj_{n,\omega} \circ \prj_{n,\omega,p_{0}}, \EQ_{G^{p_{0}}} \rangle
  = \langle \val{P(G)} \circ \prj_{n,\omega}, \EQ_{G} \rangle \circ
  \prj_{P}$.  In order to conclude the argument, observe that
  the parameterized initial algebra of a functor of the form $F \circ
  \prj_{\Cat{C}\times \Cat{D}}:\Cat{C}\times\Cat{D}\times\Cat{E} \rTo
  \Cat{C}$ has the form $\dgmu{F} \circ \prj_{\Cat{D}}$, where $\dgmu{F} :
  \Cat{D}\rTo \Cat{C}$ is the parameterized initial algebra of
  $F:\Cat{C}\times \Cat{D}\rTo \Cat{C}$.  
\end{proof}
\begin{proof}[Proof of proposition \ref{prop:termstogames}]
  By induction on the structure of $\mu$-terms.
  
  For the $\mu$-term $x$ in context $X$, we let $G$ be the parity game
  of height $0$ on the set of positions $X$, with distinguished
  position  $x \in X$.
  
  We analyze now the case of a term of the form $\Land[I] s$. By
  duality, we implicitly analyze the case of a term of the form
  $\Lor[I] s$.

  
  We first show that given a parity game $G = \langle S,\hg,\kappa,
  \epsilon \rangle$ and a subset $I \subseteq P$ of positions, it is
  possible to find a pointed parity game $\langle G', p_{0} \rangle$
  such that $\sval{G'}_{p_{0}} \iso \prod_{i \in I} \sval{G}_{i}$.  We
  define $G' = \langle S',\hg',\kappa', \epsilon' \rangle$ as follows:
  $S'$ is obtained from $S$ by adding a new position $p_{0}$ and moves
  $p_{0} \rightarrow i$ for each $i \in I$, $\hg'(p_{0}) = \height{G}
  + 1$ and $\hg'(p) = \hg(p)$ otherwise, $\kappa'(\height{G} + 1) =
  \mu$, and $\kappa'(j) = \kappa(j)$ if $j \leq \height{G}$,
  $\epsilon'(p_{0}) = \pi$ and $\epsilon'(p) = \epsilon(p)$ otherwise.
  We could also have set $\kappa'(\height{G} + 1) = \nu$,
  leading to an equivalent construction.

  Observe that $P(G') = G^{p_{0}}$, therefore
  \begin{eqnarray*}
    \val{P(G')} \circ \prj_{\Cat{C}^{P'_{\omega}}} 
    & = &
    \val{G^{p_{0}}} \circ \prj_{\Cat{C}^{\{ p_{0} \}}\times \Cat{C}^{P_{\omega}}} \\
    & \iso &
    \val{G} \circ \prj_{\Cat{C}^{P_{\omega}}} \circ \prj_{\Cat{C}^{\{ p_{0} \}}\times
      \Cat{C}^{P_{\omega}}}  \\
    & \iso & \val{G} \circ \prj_{\Cat{C}^{P_{\omega}}}\circ
    \prj_{\Cat{C}^{P_{<\omega}}\times \Cat{C}^{P_{\omega}}}\,, 
  \end{eqnarray*}
  and remark that an initial algebra for $\val{G} \circ
  \prj_{\Cat{C}^{P_{\omega}}}$ is exactly $\val{G}$.  Similarly
  \begin{eqnarray*}
    \mathcal{E}_{p_{0}}  & = &
    \prod \circ \prj(\cod,p_{0}) \\
    & \iso & (\prod_{i \in I}) \circ
    \prj_{\Cat{C}^{P_{<\omega}} \times \Cat{C}^{P_{\omega}}}\,.
  \end{eqnarray*}
  Using proposition \ref{prop:bekic2} (switch the roles of $F$ and
  $G$), compute an initial algebra of a functor of the form $\langle F
  \circ \prj_{\Cat{C}\times\Cat{E}}, G \circ
  \prj_{\Cat{C}\times\Cat{E}} \rangle: \Cat{C} \times \Cat{D} \times
  \Cat{E}\rTo \Cat{C} \times \Cat{D}$ as $\langle \dgmu{F}, G\circ
  \dgmu{F} \rangle$, $\dgmu{F}$ being the initial algebra of $F$.  In this
  formula let $F$ be $\val{G} \circ \prj_{\Cat{C}^{P_{\omega}}}$ and
  $G$ be $(\prod_{i \in I})$. It follows that $\sval{G'}_{p_{0}}$, the
  right projection in this formula, is $\sval{G'}_{p_{0}} =
  \prj_{p_{0}}\circ \val{G'} \iso \prod_{i \in I} \circ \prj_{i} \circ
  \val{G} = \prod_{i \in I} \sval{G}_{i}$.
  
  We come back to the original problem of finding a representation of
  the functor $\Land[I] s$ as a parity functor. Observing that we have
  solved the case of representing $\Land[\emptyset]$ in the previous
  discussion, we can suppose without loss of generality that $I =
  \{l,r\}$. Let $\langle G^{l},p^{l}\rangle$, $\langle G^{r},p^{r}
  \rangle$ be two pointed parity games representing $s_{l}$ and
  $s_{r}$ respectively.  Hence $G^{l}$ and $G^{r}$ share the same set
  of positions at infinity $P_{\omega} = X$.  Because of lemmas
  \ref{lemma:normalized} and \ref{lemma:suspensions}, we can assume
  that $\height{G^{l}} =\height{G^{r}} = n$ and that $\kappa(i) = \mu$
  if and only if $i$ is odd for each $i = 1,\ldots ,n$. Given these
  assumptions, we can construct a game $\lcotuple G^{l},
  G^{r}\rcotuple$ of height $n$, having as set of positions the
  disjoint union of the sets $P_{\omega}, P^{l}_{< \omega}, P^{r}_{<
    \omega}$, by pasting together the local structures of $G^{l}$ and
  $G^{r}$.  Recall that, for a pair of functors $F: \Cat{C}\times
  \Cat{E} \rTo \Cat{C}$ and $G : \Cat{D}\times \Cat{E}\rTo \Cat{D}$, a
  pair of initial algebras $(\dgmu{F},\struct{x})$ and
  $(\dgmu{G},\struct{y})$ gives rise to the algebra
  $(\langle\dgmu{F},\dgmu{G}\rangle, \langle
  \struct{x},\struct{y}\rangle)$ of the functor $\langle F\circ
  \prj_{\Cat{C}\times \Cat{E}},G \circ \prj_{\Cat{D}\times
    \Cat{E}}\rangle: \Cat{C} \times \Cat{D} \times \Cat{E}\rTo \Cat{C}
  \times \Cat{D}$, which is moreover an initial one.  Then it is easily
  verified that the relation
  \begin{eqnarray*}
    \val{\lcotuple G^{l}, G^{r}\rcotuple}
    & \iso & \ltuple \val{G^{l}} ,\val{G^{r}}\rtuple
    : \Cat{C}^{P_{\omega}} \rTo 
    \Cat{C}^{P^{l}_{< \omega}} \times \Cat{C}^{P^{r}_{< \omega}}
  \end{eqnarray*}
  holds. In this way we have reduced the problem of finding a
  representation of the $\mu$-functor $\val{s_{l} \land s_{r}}$ to the
  problem of finding a representation of the functor $\prod_{i \in
    \{p^{l},p^{r}\}} \sval{\lcotuple G^{l}, G^{r}\rcotuple}_{i}$ by a
  pointed parity game, which we have previously solved. Figure
  \ref{fig:land} displays the construction of the pointed parity
  game associated to $s_{l} \land s_{r}$.
  \begin{figure}[h] 
    $$
    \mygame{
      []!g{X}="X"
      [d]!O{p_{0}}
      (
      :[ld]
      [d(0.5)]!g{}="LGost1"
      [d]!g{\vdots}="LGost2"
      [d]!g{}="LGost3",
      :[rd]
      [d(0.5)]!g{}="RGost1"
      [d]!g{\vdots}="RGost2"
      [d]!g{}="RGost3"
      )
      "LGost2"[l(0.5)]="StartBackl"
      "RGost2"[r(0.5)]="StartBackr"
      "StartBackl":@`{"StartBackl"+(-1,1),"StartBackl"+(0,3)}"X"
      "StartBackr":@`{"StartBackr"+(1,1),"StartBackr"+(0.5,4)}"X"
      "X"("X"!L{}{\omega})
      "p_{0}"("p_{0}"!L{\mu}{2k+1})
      "LGost1"("LGost1"!L{\nu}{2k})
      "LGost3"("LGost3"!L{\mu}{1})
      "RGost1"("RGost1"!L{\nu}{2k})
      "RGost3"("RGost3"!L{\mu}{1})
    }
    $$
    \vspace{-5mm}
    \caption{Pointed parity game for $s_{l} \land s_{r}$.}
    \label{fig:land}
  \end{figure}  

  Finally, we analyze the case of a term $\mu_{x}.s$. By duality, we
  implicitly analyze the case of a term of the form $\nu_{x}.s$.
  
  Let $\langle G,p_{0} \rangle$ be a parity game such that
  $\sval{G}_{p_{0}} \iso \val{s}$.  Define the game $G' = \langle S' ,
  \hg',\kappa',\epsilon' \rangle$ as follows:
  \begin{itemize}
  \item $S'$ is obtained from  $S$ by adding the move $x
    \rightarrow p_{0}$.
  \item $\hg'(x) = \height{G} + 1$ and $\kappa'(\height{G} +
    1) = \mu$, otherwise $\hg'(p) = \hg(p)$ and $\kappa'(i) =
    \kappa(i)$ if $p \neq x$ and $i \leq \height{G}$.
  \item $\epsilon'(x) = \sigma$ and $\epsilon'(p) =
    \epsilon(p)$ if $p \neq x$.
  \end{itemize}
  Observe that $P(G') = G$ and recall that $\sval{G'}_{x}$ is the
  parameterized initial algebra of the functor in the top composite of
  the diagram below:
  $$
  \mydiagram[6em]{ []( 
    !S 
    {\Cat{C}\times\Cat{C}^{X \setminus
        \{x\}}}
    {\Cat{C}^{P_{<\omega}}\times\Cat{C}^{X}}
    {\Cat{C}^{P_{<\omega}}\times\Cat{C}^{X}}{\Cat{C}} {1}{1.7}, 
    !A {\sval{G}}{ \sval{P(G')}}
    {\prj_{p_{0}}}{\prj(\cod,x)} ) 
    "4":[r]*+{\Cat{C}}^{\coprod}
    "3" :@/^1em/"\Cat{C}"^{\mathcal{E}_{x}} 
  }
  $$
  Thus deduce that $\sval{G'}_{x}$ is also the initial algebra of
  the functor $\coprod \circ \sval{G}_{p_{0}}$. Since this functor is
  naturally isomorphic to $\sval{G}_{p_{0}}$ and therefore to
  $\val{s}$, we obtain the relation $\val{\mu_{x}.s} :=
  \sval{G'}_{x}$.
  \end{proof}




