The main result of this paper is the combinatorial characterization of
the functors on the category of sets and functions that are definable
by means of $\mu$-terms. This characterization leads to show that the
algebra of $\mu$-bicomplete categories, when realized in the category
of sets, is closely related to the theory of automata recognizing
infinite objects. For example an automaton recognizing -- by parity
condition -- infinite strings over the finite alphabet $\Sigma$ can be
described as a triple $\langle G,p,f \rangle$, where $\langle G,p
\rangle$ is a pointed parity game such that $P_{\omega} = \emptyset$,
$\epsilon(p) = \sigma$ for all $p \in P$, and $f : \sval{G}_{p} \rTo
\val{\nu_{x}.\bigvee_{\Sigma} x}$ is a function arising from labeling
the transitions of $G$ by symbols in $\Sigma$, function which turns
out to be definable in the language of $\mu$-bicomplete categories. A
subset $L \subseteq \Sigma^{\nnumbers}$ is recognizable if and only if
there exists such a triple $\langle G,p,f \rangle$, so that $L$ is the
image of $f$.  A main motivation for developing this work was indeed
to make available to this theory an algebraic language (the one of
$\mu$-bicomplete categories) which is alternative but also analogous
to the one of $\mu$-calculi \cite{AN01}.

The combinatorial characterization suggests also a way for enlarging
the collection of categories which are known to be $\mu$-bicomplete.
There are several toposes that occur in computer science -- for
example, the effective topos \cite{hyland} -- which are not complete
or cocomplete, in particular they are neither locally presentable nor
dually locally presentable.  A detailed analysis of the work presented
here could show that the explicit characterization of parity functors
can be carried within intuitionistic logic. If this were the case, the
characterization could be used to show that elementary toposes with a
natural number object are $\mu$-bicomplete.

Finally, it is an open problem to understand whether this
game-theoretic characterization is useful to understand $\mu$-functors
in arbitrary categories.  It is in general easier to understand
several algebraic equivalences in terms of game equivalences. We have
avoided to make precise this notion, but we conjecture that this can
be done so that two parity games are game-theoretic equivalent if and
only if their interpretations as functors are naturally isomorphic in
every $\mu$-bicomplete category.

