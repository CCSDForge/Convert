
We define $\mu$-bicomplete categories by mimicking the definition of
$\mu$-algebras \cite{MR87e:03079} at the level of categories:
$\mu$-terms are defined and an algebra is a $\mu$-algebra if it is
possible to interpret all the $\mu$-terms as expected.  In a
categorical context a $\mu$-term is to be interpreted as a functor,
which generalizes the usual interpretation of a $\mu$-term as an order
preserving function.

In the following definition we explicitly keep track of free variables
in a $\mu$-term by means of a context $X$: this is simply a finite set
(of variables). Later we shall use the notation $\Cat{C}^{X}$ to
denote the $X$-fold product of a category $\Cat{C}$ with itself.

\begin{dfntn}
  The set $\mu\mathcal{T}(X)$ of $\mu$-terms over a context $X$ is
  defined as follows:
  \begin{enumerate} 
  \item For each pair $(X,x)$, where $X$ is a finite set and $x \in
    X$, $x \in \mu\mathcal{T}(X)$.
  \item If $I$ is a finite set and $s : I \rTo \mu\mathcal{T}(X)$,
    then $\Land[I] s, \Lor[I] s \in  \mu\mathcal{T}(X)$.
  \item If $s \in \mu\mathcal{T}(X)$ and $x \in X$, then $\mu_{x}.s,
    \nu_{x}.s \in \mu\mathcal{T}(X \setminus \{ x \})$.
  \end{enumerate}
\end{dfntn}

\begin{dfntn}
  Let $\Cat{C}$ be a category with finite products and finite
  coproducts. We define a partial interpretation of $\mu$-terms $s \in
  \mu\mathcal{T}(X)$ over a context $X$ as functors of the form
  $\val{s}: \Cat{C}^{X} \rTo \Cat{C}$.
  \begin{enumerate}
  \item For $x \in X$, we let $\val{x} = \prj_{x}:
    \Cat{C}^{X} \rTo \Cat{C}$.
  \item We let $\val{\Land[I] s} = \prod_{i \in I} \val{s_{i}}$ and
    $\val{\Lor[I] s} = \coprod_{i \in I} \val{s_{i}}$, given that all
    the $\val{s_{i}}$ are defined.
  \item We let $\val{\mu_{x}.s}$ be the parameterized initial algebra
    of
    $$
    \val{s}: \Cat{C}^{\{x\}} \times \Cat{C}^{X \setminus \{x\}}
    \rTo \Cat{C}\,,
    $$
    given that $\val{s}$ is defined. Similarly we let
    $\val{\nu_{x}.s}$ be the parameterized final coalgebra of
    $\val{s}$. If $\val{s}$ is not defined or if the desired initial
    algebras (final coalgebras) do not exist, then we leave
    $\val{\mu_{x}.s}$ ($\val{\nu_{x}.s}$) undefined.
  \end{enumerate}
\end{dfntn}

\begin{dfntn}
  A category with finite products and finite coproducts $\Cat{C}$ is
  said to be \emph{$\mu$-bicomplete} if for each finite set of
  variables $X$ and $\mu$-term $s \in \mu\mathcal{T}(X)$ the
  interpretation $\val{s}$ is defined.
\end{dfntn}

An alternative point of view emphasizes the class of functors which
are definable by means of $\mu$-terms in a $\mu$-bicomplete category.
Thus we are lead to give the following definition.

\begin{dfntn}
  We say that a functor $F : \Cat{C}^{X} \rTo \Cat{C}^{Y}$ is a
  \emph{$\mu$-functor} if there exist a collection of $\mu$-terms $\{
  s_{y} \in \mu\mathcal{T}(X) \}_{y \in Y}$
  and a natural isomorphism $F \iso \ltuple \val{s_{y}} \rtuple_{y \in
    Y}$.
\end{dfntn}

\begin{prpstn}
  $\mu$-functors are closed under composition.
\end{prpstn}
\begin{proof}
  Let $s \in \mu\mathcal{T}(X)$ and $z \not\in X$, we first define
  $s^{z} \in \mu\mathcal{T}(\{z \} \cup X)$ with the property that
  $\val{s^{z}} = \val{s}\circ \prj_{X}$, by induction on the structure
  of $s$.  We let $x^{z} = x$, $(\Land[I] s)^{z} = \Land[I] s^{z}$ and
  $(\Lor[I] s)^{z} = \Lor[I] s^{z}$, where $(s^{z})_{i} =
  (s_{i})^{z}$, $(\mu_{x}.s)^{z} = \mu_{x}.(s^{z})$ and
  $(\nu_{x}.s)^{z} = \nu_{x}.(s^{z})$. In the last two cases we have
  supposed that the variable $x \not\in \{z\} \cup X$, otherwise we
  can rename $x$ in $s$ to a variable $x' \not\in \{z\}\cup X $ and
  obtain a $\mu$-term $t \in \mu\mathcal{T}(\{x'\} \cup X)$ such that
  $\val{t} = \val{s}$ and $\val{\mu_{x'}.t} = \val{\mu_{x}.s}$ and
  then we can define $(\mu_{x}.s)^{z} = \mu_{x'}.(t^{z})$.
  
  Let $s : Y \rTo \mu\mathcal{T}(X)$ be a collection of $\mu$-terms
  and let $t \in \mu\mathcal{T}(Y)$. We define now a $\mu$-term $t[s]
  \in \mu\mathcal{T}(X)$ with the property that $\val{t[s]} \iso
  \val{t} \circ \ltuple \val{s_{y}}\rtuple_{y \in Y}$, by induction of
  the structure of $t$.  We let $y[s] = s_{y}$, $(\Land[I] t)[s] =
  \Land[I] (t[s])$, $(\Lor[I] t)[s] = \Lor[I] (t[s])$ where $t[s]_{i}
  = t_{i}[s]$.  Eventually, we let $(\mu_{x}.t)[s] =
  \mu_{x}.(t[x,s^{x}])$ and $(\nu_{x}.t)[s] 
    = \nu_{x}.(t[x,s^{x}])$, where $(x,s^{x}): \{x\}\cup Y \rTo
  \mu\mathcal{T}(\{x \} \cup X)$ is such that $(x,s^{x})_{y} =
  s_{y}^{x}$ if $y \in Y$ and $(x,s^{x})_{x} = x$.  We have supposed
  again and without loss of generality that $x \not\in X$. The desired
  statement follows.
\end{proof}

Let $Y$ be a set of variables and suppose that it is the disjoint
union of $X$ and $Z$.  We can extend every collection $s : Z \rTo
\mu\mathcal{T}(X)$ indexed by $Z$ to a collection $s' : Y \rTo
\mu\mathcal{T}(X)$ by letting $s'_{y} = s_{y}$ if $y \in Z$ and
$s'_{y} = y$ if $y \in X$. Thus if $t \in \mu\mathcal{T}(Y)$ then we
let
\begin{eqnarray*}
  t[s_{z}/z]_{z\in Z} & = & t[s']
\end{eqnarray*}
where $t[s']$ has been defined in the proof of the above proposition.
We observe that the interpretation of $\val{t[s_{z}/z]_{z\in Z}}$ is
$$
\Cat{C}^{X} \rTo[l>=8em]^{\ltuple \,\ltuple \val{s_{z}} \rtuple_{z
    \in Z}\,,\,\Id_{\Cat{C}^{X}}\,\rtuple} \Cat{C}^{Z} \times
\Cat{C}^{X} \rTo[l>=4em]^{\val{t}} \Cat{C}
$$
according to the previous proposition.


\begin{prpstn}
  $\mu$-functors are closed under parameterized initial algebras and
  parameterized final coalgebras.
\end{prpstn}

By this property we mean that if $F: \Cat{C}^{Y} \rTo \Cat{C}^{X}$ is
a $\mu$-functor and $X \subseteq Y$, so that we can represent
$\Cat{C}^{Y}$ as the product $\Cat{C}^{X} \times \Cat{C}^{Y \setminus
  X}$, then we can find a collection of $\mu$-terms $\{ t_{x} \in
\mu\mathcal{T}(Y \setminus X)\}_{x \in X}$ so that $\ltuple
\val{t_{x}}\rtuple_{x \in X}: \Cat{C}^{Y \setminus X} \rTo
\Cat{C}^{X}$ is a parameterized initial algebra of $F$ and, similarly,
it is possible to find an analogous representation for a parameterized
final coalgebra of $F$.  The proposition is an immediate consequence
of the Beki\v{c} property and its dual for final coalgebras. It will
also be evident from the representation of $\mu$-functors by means of
parity functors that we describe in the next section.



We want to find concrete examples of $\mu$-bicomplete categories.  To
achieve this goal, we shall look at locally presentable categories
which, in some sense, generalize complete lattices. We briefly recall
the principal concepts that define these categories, their properties
being described in the monographs \cite{makkaipare,lpac}.

Let $\lambda$ be a regular cardinal. A poset is $\lambda$-directed if
every subset of cardinality less than $\lambda$ has an upper bound. If
$D : J \rTo \Cat{C}$ is a diagram whose index $J$ is a
$\lambda$-directed poset, then we say that $D$ is $\lambda$-directed
and that its colimit, whenever it exists, is $\lambda$-directed.  A
functor $T: \Cat{C} \rTo \Cat{D}$ is said to be $\lambda$-accessible
if it preserves $\lambda$-directed colimits.  An object $c$ of a
category $\Cat{C}$ is $\lambda$-presentable if the hom-functor
$\Cat{C}(c,-):\Cat{C} \rTo \Sets$ is $\lambda$-accessible.  Thus: a
category $\Cat{C}$ is locally $\lambda$-presentable if (1) it is
cocomplete and (2) it has a set $\mathcal{A}$ of $\lambda$-presentable
objects such that every object of $\Cat{C}$ is the $\lambda$-directed
colimit of objects from $\mathcal{A}$. We can relax condition (1) to:
(1') it has all the $\lambda$-directed colimits, in which case
conditions (1') and (2) define a $\lambda$-accessible category.
Finally: a functor is said to be \emph{accessible} if it is
$\lambda$-accessible for some regular cardinal $\lambda$.  A category
is said to be \emph{locally presentable} (\emph{accessible}) if it is
locally $\lambda$-presentable ($\lambda$-accessible) for some regular
cardinal $\lambda$.



  

Most of the common
categories are locally presentable: the category of sets and
functions, categories of presheaves and sheaves, varieties and
quasivarieties of algebras. Thus, in the rest of this section, we
shall prove:
\begin{thrm}
  \label{theo:lpc}
  Every locally presentable category is $\mu$-bicomplete.
\end{thrm}

In order to show that a category $\Cat{C}$ is $\mu$-bicomplete, it
suffices to find a class of functors of the form $\Cat{C}^{J} \rTo
\Cat{C}$, where $J$ ranges on finite sets, that contains the
projections and is closed under finite products, finite coproducts,
and formation of parameterized initial algebras and parameterized
final coalgebras. We recall the following facts about accessible
functors:
\begin{itemize}
\item Left and right adjoints 
  between accessible categories are accessible \cite[\S 2.23]{lpac}.
\item If $\Cat{D}$ has an initial and a final object, then a
  projection $\prj_{\Cat{C}}: \Cat{C}\times \Cat{D} \rTo \Cat{C}$ is
  both a left and a right adjoint.
\item Coproducts, diagonals and products are adjoints, since $\coprod
  \dashv \Delta \dashv \prod: \Cat{C}^{J} \rTo \Cat{C}$.
\end{itemize}
Knowing that locally presentable categories are closed under finite
products, we conclude that if $\Cat{C}$ is such a
category, then the class of accessible functors $F: \Cat{C}^{J} \rTo
\Cat{C}$ contains the projections and is closed under finite products
and finite coproducts.
It is well known that initial algebras and final coalgebras of
$\lambda$-accessible unary functors exist in locally presentable
categories \cite{MR82d:18004,barr}; moreover if $F: \Cat{C}\times
\Cat{D} \rTo \Cat{C}$ is $\lambda$-accessible, so is the unary functor
$F(-,d):\Cat{C} \rTo \Cat{C}$ for each object $d$ of $\Cat{D}$. Thus,
in order to conclude that locally presentable categories are
$\mu$-bicomplete, we need the following proposition:
\begin{prpstn}
  \label{lemma:closed}
  If $\Cat{C}$ and $\Cat{D}$ are locally presentable categories and $F
  :
  \Cat{C}\times \Cat{D} \rTo \Cat{C}$ is an accessible functor, then
  both the parameterized initial algebra $\dgmu{F} : \Cat{D} \rTo
  \Cat{C}$ and the parameterized final coalgebra $\dgnu{F} : \Cat{D}
  \rTo \Cat{C}$ are accessible.
\end{prpstn}
We are thankful to Alex Simpson for pointing out the following short
proof that relies on general properties of locally presentable
categories and accessible functors.
\begin{proof}
  We only prove that $\dgnu{F}$ is accessible, since the proof for
  $\dgmu{F}$ is dual.  Consider the category $\Cat{E}$ with objects
  $(c,d,\zeta)$, where $c \in \Cat{C}$, $d \in \Cat{D}$, and $\zeta: c
  \rTo F(c,d)$, and with morphisms $(f,g):(c,d,\zeta) \rTo
  (c',d',\zeta')$ being maps $f:c \rTo c'$ and $g:d \rTo d'$ such that
  $\zeta \comp F(f,g) = f \comp \zeta'$. Observe that there is an
  obvious forgetful functor $\Cat{E} \rTo \Cat{C}\times\Cat{D}$ as
  well as a natural transformation
  $$
  \myspecialdiagram{25}{
    []*+{\Cat{E}}="E"
    ([r(1.5)u(0.4)]*+{\Cat{C}\times\Cat{D}}="P1",
     [r(1.5)d(0.4)]*+{\Cat{C}\times\Cat{D}}="P2")
    [r(3)]*+{\Cat{C}}="C"
    "E":@/^0.5em/"P1"
    "E":@/_0.5em/"P2"
    "P2":@/_0.5em/"C"_{F}_{}="A"
    "P1":@/^0.5em/"C"^{\prj_{\Cat{C}}}_{}="B"
    "P1"[d(0.2)]="P1"
    "P2"[u(0.2)]="P2"
    "P1":@2"P2"_{\zeta}
  }
  $$
  The 2-categorical diagram above is the inserter -- cf. \cite[\S
  4]{kelly} -- of $\prj_{\Cat{C}}$ and $F$ and this implies that
  $\Cat{E}$ is accessible, since accessible categories are closed
  under lax limits \cite[\S 5.1.8]{makkaipare}. Also, it is easily
  verified that the forgetful functor $\Cat{E} \rTo
  \Cat{C}\times\Cat{D}$ creates colimits, so that $\Cat{E}$ is
  cocomplete, hence locally presentable.
  
  There is a functor $G: \Cat{D} \rTo \Cat{E}$ mapping an object $d$
  of $\Cat{D}$ to $(\dgnu{F}d,d,\zeta_{d})$ where $\zeta_{d}:\dgnu{F}d
  \rTo F(\dgnu{F}d,d)$ is a final coalgebra. Then $G$ is right adjoint
  to the accessible functor $\Cat{E} \rTo \Cat{C}\times \Cat{D} \rTo
  \Cat{D}$, hence $G$ is accessible. But $\dgnu{F}$ is simply
  $$
  \mydiagram[6em]{
    []*+{\Cat{D}}
    :[r]*+{\Cat{E}}^{G}
    :[r]*+{\Cat{C}\times\Cat{D}}
    :[r]*+{\Cat{C}}^{\prj_{\Cat{C}}}
  }
  $$
  which, as a composite of accessible functors, is accessible. 
\end{proof}

\begin{finalita}
  It is possible to directly prove proposition \ref{lemma:closed}
  along the lines of \cite{barr}. Such a proof also shows that if
  $\Cat{C}$ is a locally $\lambda$-presentable category with
  $\lambda > \omega$, then the class of
  $\lambda$-accessible functors of the form $\Cat{C}^{J} \rTo \Cat{C}$
  is closed under formation of parameterized final coalgebras.  The
  condition $\lambda > \omega$ is necessary, cf. \cite{itmo}: the
  interpretation of the $\mu$-term $\nu_{y}.(x \vee (y \land y))$ in
  the category of sets is the functor that associates to each set $X$
  the set of infinite binary trees with leaves labeled in $X$. Letting
  $X$ be the set $\nnumbers$ of natural numbers, we observe that there
  are infinitely many binary trees whose leaves are labeled by an
  infinite subset of $\nnumbers$, thus the set of this infinite binary
  tree is not the inductive limit of the sets of infinite trees whose
  leaves are labeled by a finite subset of $\nnumbers$.  Since
  $\nnumbers$ is the inductive limit of its finite subsets, we see
  that this functor is not $\omega$-accessible.  Finally, since finite
  products are $\lambda$-accessible in locally $\lambda$-presentable
  categories \cite[\S1.59]{lpac}, we can infer:
  \begin{prpstn}
    If $\lambda > \omega$, then every $\mu$-functor on a locally
    $\lambda$-presentable category is $\lambda$-accessible.
  \end{prpstn}
\end{finalita}

\begin{labripreprint} 
  We are going to give an alternative proof of proposition
  \ref{lemma:closed}, along the lines of \cite{barr}.  The proof that
  follows will show in particular that if $\Cat{C}$ is a locally
  $\lambda$-presentable category, where $\lambda > \omega$ is a
  regular cardinal, then the class of $\lambda$-accessible functors of
  the form $\Cat{C}^{J} \rTo \Cat{C}$ is closed under formation of
  parameterized final coalgebras.  The condition $\lambda > \omega$ is
  necessary, cf. \cite{itmo}: the interpretation of the $\mu$-term
  $\nu_{y}.(x \vee (y \land y))$ in the category of sets is the
  functor that associates to each set $X$ the set of infinite binary
  trees with leaves labeled in $X$. Letting $X$ be the set $\nnumbers$
  of natural numbers, we observe that there are infinitely many binary
  trees whose leaves are labeled by an infinite subset of $\nnumbers$,
  thus the set of this infinite binary tree is not the inductive limit
  of the sets of infinite trees whose leaves are labeled by a finite
  subset of $\nnumbers$.  Since $\nnumbers$ is the inductive limit of
  its finite subsets, we see that this functor is not
  $\omega$-accessible.
  
  Since it is well known that finite products are $\lambda$-accessible
  in locally $\lambda$-presentable categories, cf. for example
  \cite[\S1.59]{lpac}, this result will also imply the following fact:
  \begin{prpstn}
    If $\lambda > \omega$, then every $\mu$-functor on a locally
    $\lambda$-presentable category is $\lambda$-accessible.
  \end{prpstn}

  To prove proposition
  \ref{lemma:closed}, we rely on standard properties and
  language of locally presentable categories, cf.  \cite{lpac}.
  We begin from initial algebras.
  \begin{prpstn}
    Let $\Cat{C},\Cat{W}$ be a locally $\lambda$-presentable categories
    and $T: \Cat{C} \times \Cat{W} \rTo \Cat{C}$ be a
    $\lambda$-accessible functor. Then, for each object $w$ of $\Cat{W}$
    an initial $T(-,w)$-algebra $(\bd{x}_{w},\struct{x}_{w})$ exists and
    the induced functor $\bd{x} : \Cat{W} \rTo \Cat{C}$ is again
    $\lambda$-accessible.
  \end{prpstn}
  \begin{proof}
    It is well known that such an initial algebra exists, see for
    example \cite{MR82d:18004}.  Let $I$ be a $\lambda$-directed poset,
    $w: I \rTo \Cat{W}$ a functor with colimiting cocone
    $$
    (\inj_{w_{i}}:w_{i} \rTo w)_{i \in I}\,.
    $$
    We shall show first that $\colim \bd{x}_{w_{i}}$ has a
    $T(-,w)$-algebra structure which is moreover initial.  Recall that
    the arrow
    $$
    \lcotuple T(\inj_{\bd{x}_{w_{i}}},\inj_{w_{i}})\rcotuple : \colim
    T(\bd{x}_{w_{i}},w_{i}) \rTo T(\colim \bd{x}_{w_{i}},w)
    $$
    is invertible, since $T$ is $\lambda$-accessible, and thus
    construct a $T(-,w)$-algebra as follows:
    $$
    T(\colim \bd{x}_{w_{i}},w) \iso \colim T(\bd{x}_{w_{i}},w_{i})
    \rTo[l>=4em]^{\colim \struct{x}_{w_{i}}} \colim \bd{x}_{w_{i}}\,.
    $$
    Let $\alpha: T(a,w) \rTo a$ be a $T(-,w)$-algebra, and, for each
    $i \in I$, form the $T(-,w_{i})$-algebra
    $$
    T(a,\inj_{w_{i}}) \comp \alpha : T(a,w_{i}) \rTo T(a,w) \rTo a\,.
    $$
    Let $f_{i}: \bd{x}_{w_{i}} \rTo a$ be such that
    \begin{eqnarray*}
      \struct{x}_{w_{i}} \comp f_{i} 
      & = & 
      T(f_{i},w_{i}) \comp T(a,\inj_{w_{i}}) \comp \alpha\,,
    \end{eqnarray*}
    and argue that $f_{i} = \bd{x}_{w_{ij}}\comp f_{j}$ -- i.e.
    $(f_{i}:\bd{x}_{w_{i}} \rTo a)_{i \in I}$ is a cocone -- by means
    of the following diagram
    $$
    \mydiagrambot[6em]{ []( !S
      {T(\bd{x}_{w_{i}},w_{i})}{\bd{x}_{w_{i}}}
      {T(\bd{x}_{w_{j}},w_{j})}{\bd{x}_{w_{j}}} {1}{1.5}, !A
      {\struct{x}_{w_{i}}}{T(\bd{x}_{w_{ij}},w_{ij})}
      {\bd{x}_{w_{ij}}}{} "1"="Start" ) "3" ( !S
      {T(\bd{x}_{w_{j}},w_{j})}{\bd{x}_{w_{j}}} {T(a,w_{j})}{a}
      {1}{1.5}, !A {\struct{x}_{w_{j}}}{T(f_{j,w_{j}})}
      {f_{j}}{T(a,\inj_{w_{j}})\comp \alpha} "3"="T(a,w_{j})" ) "Start"
      :[ru]*+{T(a,w_{i})}^{T(\bd{x}_{w_{ij}} \comp f_{j},w_{i})}
      :"T(a,w_{j})"^{T(a,w_{ij})} }\,.
    $$
    The relation
    \begin{eqnarray*}
      \colim \struct{x}_{w_{i}} \comp \lcotuple f_{i}
      \rcotuple
      & = & \lcotuple T(\inj_{\bd{x}_{w_{i}}},\inj_{w_{i}})  \rcotuple
      \comp 
      T(\lcotuple f_{i} \rcotuple, a) \comp \alpha\,,
    \end{eqnarray*}
    is easily deduced as follows:
    \begin{eqnarray*}
      \lefteqn[3cm]{\inj_{T(\bd{x}_{w_{i}},w_{i})} \comp \colim \struct{x}_{w_{i}} \comp \lcotuple f_{i}
        \rcotuple}\\
      & = & \struct{x}_{w_{i}} \comp f_{i} \\
      & = & T(f_{i},w_{i})\comp T(a,\inj_{w_{i}}) \comp \alpha \\
      & = & 
      T(\inj_{\bd{x}_{w_{i}}} \comp \lcotuple f_{i} \rcotuple,\inj_{w_{i}}) \comp \alpha \\
      & = & 
      T(\inj_{\bd{x}_{w_{i}}},\inj_{w_{i}}) \comp 
      T(\lcotuple f_{i} \rcotuple, w) \comp \alpha \\
      & = & \inj_{T(\bd{x}_{w_{i}},w_{i})} \comp 
      \lcotuple T(\inj_{\bd{x}_{w_{i}}},\inj_{w_{i}})  \rcotuple
      \comp 
      T(\lcotuple f_{i} \rcotuple, w) \comp \alpha
    \end{eqnarray*}
    and hence we obtain
    \begin{eqnarray}
      \label{eq:goal}
      \lcotuple T(\inj_{\bd{x}_{w_{i}}},\inj_{w_{i}})  \rcotuple^{-1}
      \comp \colim \struct{x}_{w_{i}} \comp \lcotuple f_{i}
      \rcotuple
      & = & 
      T(\lcotuple f_{i} \rcotuple, w) \comp \alpha\,.
    \end{eqnarray}
    On the other hand, if a relation like $(\ref{eq:goal})$ with $g$ in
    place of $\lcotuple f_{i} \rcotuple$ holds, then we obtain
    \begin{eqnarray*}
      \struct{x}_{w_{i}}\comp \inj_{\bd{x}_{w_{i}}} \comp g
      & = & 
      T(\inj_{\bd{x}_{w_{i}}},\inj_{w_{i}}) \comp  
      T(g, w) \comp \alpha \\
      & = & 
      T(\inj_{\bd{x}_{w_{i}}} \comp g,w_{i}) \comp  
      T(a,\inj_{w_{i}}) \comp \alpha
    \end{eqnarray*}
    for each $i\in I$. Therefore $\inj_{\bd{x}_{w_{i}}} \comp g =
    f_{i}$ and $g = \lcotuple f_{i} \rcotuple$.
    
    In order to conclude the argument, observe that for the particular
    case of the given initial algebra $\struct{x}_{w}:T(\bd{x}_{w},w)
    \rTo w$, the $f_{i}$'s are the $\bd{x}_{\inj_{w_{i}}}$'s, since
    $\struct{x}_{w_{i}} \comp \bd{x}_{\inj_{w_{i}}} =
    T(\bd{x}_{\inj_{w_{i}}},w_{i}) \comp T(\bd{x}_{w},\inj_{w_{i}})
    \comp \struct{x}_{w}$. It follows that $\lcotuple
    \bd{x}_{\inj_{w_{i}}} \rcotuple$ is a morphism of
    $T(-,w)$-algebras between two initial ones, and henceforth it is
    invertible.
  \end{proof}
  
  We analyze next parameterized final coalgebras. 
  \begin{prpstn}
    Let $\lambda > \omega$ be a regular cardinal and $T: \Cat{C} \rTo
    \Cat{C}$ be a $\lambda$-accessible functor of a locally
    $\lambda$-presentable category.  Call $\mathcal{A}_{T}$ the class of
    $T$-coalgebras $(a,\alpha)$ such that $a$ is $\lambda$-presentable
    as an object of $\Cat{C}$. Then, up to isomorphism,
    $\mathcal{A}_{T}$ is a set and moreover it is a strong generator for
    the category of $T$-coalgebras $\Cat{C}_{T}$.
  \end{prpstn}
  We observe that the above proposition does not imply that
  $\Cat{C}_{T}$ is locally $\lambda$-presentable since a coalgebra
  $(a,\alpha) \in \mathcal{A}_{T}$ need not be $\lambda$-presentable in
  the category $\Cat{C}_{T}$.
  
  \begin{proof}
    It is easily shown that $\mathcal{A}_{T}$ is, up to isomorphism, a
    set, so that we will show that each coalgebra $(c,\gamma)$ is the
    colimit of the canonical diagram $\dom: \mathcal{A}_{T}/(c,\gamma)
    \rTo \Cat{C}_{T}$.
    
    To this end, fix the coalgebra $\gamma : c \rTo Tc$, let
    $\mathcal{A}$ be the class of $\lambda$-presentable objects in
    $\Cat{C}$ and call $U_{c}$ the lifting of the forgetful functor $U :
    \Cat{C}_{T} \rTo \Cat{C}$ to the comma categories:
    $$
    \mydiagrambot[7em]{ []( !S
      {\mathcal{A}_{T}/(c,\gamma)}{\Cat{C}_{T}} {\mathcal{A}/c}{\Cat{C}}
      {1}{1}, !A {\dom}{U_{c}} {U}{\dom} ) [] }\,.
    $$
    We shall show that the functor $U_{c}$ is cofinal, from which it
    follows
    $$
    \renewcommand{\arraystretch}{1.3}
    \begin{array}[b]{rcl@{\hspace{7mm}}l}
      c & = & \colim \dom \\
      & = & \colim (\dom \circ U_{c})
      & \textrm{since $U_{c}$ is cofinal} \\
      & = & \colim (U \circ \dom)\,,
    \end{array}
    \renewcommand{\arraystretch}{1}
    $$
    and
    \begin{eqnarray*}
      (c,\gamma) & = & \colim \dom\,,
    \end{eqnarray*}
    since the functor $U : \Cat{C}_{T} \rTo \Cat{C}$ creates colimits.
    Recall that an object of $\mathcal{A}_{T}/(c,\gamma)$ is a triple
    $(a,\alpha,f)$ such that $(a,\alpha)$ is a coalgebra in
    $\mathcal{A}_{T}$ and $f: (a,\alpha) \rTo (c,\gamma)$ is a
    coalgebra morphism; an arrow $h:(a,\alpha,f) \rTo (b,\beta,g)$ of
    $\mathcal{A}_{T}/(c,\gamma)$ is a coalgebra morphism $h:(a,\alpha)
    \rTo (b,\beta)$ such that $h \comp g = f$.  The explicit
    description of the functor $U_{c}$ is as follows: the triple
    $(a,\alpha,f)$ is sent to the pair $(a,f)$ and the arrow
    $h:(a,\alpha,f) \rTo (b,\beta,g)$ is sent to $h:(a,f) \rTo (b,g)$.
    
    We begin by showing that for a given object $(a,f)$ of
    $\mathcal{A}/c$, we can find an object $(b,\beta,g)$ of
    $\mathcal{A}_{T}/(c,\gamma)$ and an arrow $h: (a,f) \rTo (b,g)$ in
    $\mathcal{A}/c$.
    
    Recall that $c = \colim b_{j}$ where $b : J \rTo \Cat{C}$ is a
    $\lambda$-directed diagram taking image in $\mathcal{A}$.  Let
    $(b_{0},g_{0}) = (a,f)$.  Suppose that we have constructed an
    object $(b_{n},g_{n})$ of $\mathcal{A}/c$, then we can construct a
    new object $(g_{n +1},b_{n+ 1})$ of $\mathcal{A}/c$ and an arrow
    $\beta_{n} : b_{n} \rTo Tb_{n +1}$ such that the diagram
    $$
    \mydiagram[6em]{ [] ( !S {b_{n}}{c} {Tb_{n + 1}}{Tc} {1}{1}, !A
      {g_{n}}{\beta_{n}} {\gamma}{Tg_{n + 1}} ) }
    $$
    is commutative. This is possible, since $b_{n}$ is
    $\lambda$-presentable, $T$ is $\lambda$-accessible, so that
    $(T\inj_{j}:Tb_{j} \rTo Tc)_{j \in J}$ is a colimiting cocone.
  
    Consider now the coalgebra
    $$
    \coprod_{n \geq 0} b_{n} \rTo[l>=4em]^{\coprod_{n \geq 0}
      \beta_{n}} \coprod_{n \geq 0} Tb_{n + 1} \rTo[l>=4em]^{\lcotuple
      T\inj_{b_{n + 1}}\rcotuple} T(\coprod_{n \geq 0} b_{n})\,.
    $$
    Its carrier $\coprod_{n \geq 0} b_{n}$ is a $\lambda$-small
    colimit of $\lambda$-presentable objects, hence it is again
    $\lambda$-presentable.  We are using here the fact that $\lambda >
    \omega$, so that a countable colimit of $\lambda$-presentable
    objects is again $\lambda$-presentable, cf. \cite[\S 1.16]{lpac}.
    
    Moreover $\lcotuple g_{n} \rcotuple$ is a morphism of coalgebras to
    $(c,\gamma)$, by virtue of the following calculations:
    \begin{eqnarray*}
      \lefteqn[3cm]{\inj_{b_{n}} \comp 
        (\coprod_{n \geq 0} \beta_{n}  
        \comp \lcotuple T\inj_{b_{n+ 1}} \rcotuple)
        \comp T\lcotuple g_{n} \rcotuple} \\[-0.8em]
      & = & \beta_{n} \comp \inj_{Tb_{n + 1}} \comp \lcotuple T\inj_{b_{n
          + 1}} \rcotuple \comp T\lcotuple g_{n} \rcotuple \\
      & = & \beta_{n} \comp  T\inj_{b_{n
          + 1}} \comp T\lcotuple g_{n} \rcotuple \\
      & = & \beta_{n} \comp  Tg_{n + 1} \\ 
      & = & g_{n} \comp \gamma \\
      & = & \inj_{b_{n}} \comp \lcotuple g_{n} \rcotuple \comp \gamma\,.
    \end{eqnarray*}
    Summarizing, let $b = \coprod_{n \geq 0} b_{n}$, $\beta =
    \coprod_{n \geq 0} \beta_{n} \comp \lcotuple T\inj_{b_{n+ 1}}
    \rcotuple$, $g = \{g_{n} \}$, $h = \inj_{b_{0}}$, then
    $(b,\beta,g)$ is the desired object of
    $\mathcal{A}_{T}/(c,\gamma)$ and $h: (a,f) \rTo (b,g)$ is the
    desired arrow of $\mathcal{A}/c$, since $\inj_{b_{0}} \comp
    \lcotuple g_{n} \rcotuple = g_{0} = f$.
  
    We show now that given two arrows $h_{i} : (a,f) \rTo
    U_{c}(b_{i},\beta_{i},g_{i})$, $i = 1,2$, we can find an object
    $(d,\delta,k)$ and two arrows $l_{i}:(b_{i},\beta_{i},g_{i}) \rTo
    (d,\delta,k)$ such that $h_{1} \comp l_{1} = h_{2} \comp l_{2}$.
    We are given the diagram
    $$
    \mydiagram[6em]{ [] ( []*+{a}="1"
      ([dl]*+{b_{1}}="2",[d]*+{c}="4",[dr]*+{b_{2}}="3") !a
      {_{h_{1}}}{^{h_{2}}} {^{g_{1}}}{_{g_{2}}}, "2"="b1", "3"="b2" )
      "b1" ( !S {b_{1}}{Tb_{1}} {c}{Tc} {1}{1}, !A {\beta_{1}}{}
      {Tg_{1}}{\gamma} ) "b2" ( !S {b_{2}}{Tb_{2}} {c}{Tc} {1}{-1}, !a
      {^{\beta_{2}}}{} {_{Tg_{2}}}{} ) }
    $$
    where $f = h_{1} \comp g_{1} = h_{2} \comp g_{2}$.
    Form the pushouts from  $a$ 
    to obtain the diagram:
    $$
    \mydiagrambot[5.5em]{ []( !S {b_{1}}{Tb_{1}} {b_{1} \push{a}
        b_{2}} {Tb_{1} \push{a} Tb_{2}} {2}{1}, !a
      {^{\beta_{1}}}{^/-2mm/{\inj_{b_{1}}}}
      {^/-3mm/{\inj_{Tb_{1}}}}{^{\beta_{1} \push{a} \beta_{2}}}, !s
      {b_{1}}{Tb_{1}} {c}{Tc} {2}{0}{2}{1}, !a {}{_/-5mm/{g_{1}}}
      {}{^/-10mm/{\gamma}} "3"="c" "4"="Tc" ) [rrr] ( !S
      {b_{2}}{Tb_{2}} {b_{1} \push{a} b_{2}} {Tb_{1} \push{a} Tb_{2}}
      {2}{-2}, !a {^{\beta_{2}}}{_{\inj_{b_{2}}}}
      {_/-5mm/{\inj_{Tb_{2}}}}{} "3"="b_{1} \push{a} b_{2}"
      "4"="Tb_{1} \push{a} Tb_{2}" , !s {b_{2}}{Tb_{2}} {c} {Tc}
      {2}{0}{-1}{1}, !A {}{g_{2}} {}{} ) "Tb_{1} \push{a} Tb_{2}" (
      :[d]*+{T(b_{1}\push{a} b_{2})}_/3mm/{\lcotuple
        T\inj_{b_{1}},T\inj_{b_{2}} \rcotuple} :"Tc"_/2mm/{T\lcotuple
        g_{1},g_{2} \rcotuple}, :"Tc"|/-2.5mm/{\lcotuple Tg_{1},Tg_{2}
        \rcotuple} ) "b_{1} \push{a} b_{2}":"c"|/-2.5mm/{\lcotuple
        g_{1},g_{2} \rcotuple} }\,.
    $$
    We let $d = b_{1}\push{a} b_{2}$, $\delta = \beta_{1} \push{a}
    \beta_{2} \comp \lcotuple T\inj_{b_{1}},T\inj_{b_{2}} \rcotuple$,
    $k =\lcotuple g_{1},g_{2} \rcotuple$, then $(d,\delta,k)$ is an
    object of $\mathcal{A}_{T}/(c,\gamma)$:
    \begin{eqnarray*}
      \delta \comp Tk & = & 
      \beta_{1} \push{a} \beta_{2} \comp 
      \lcotuple T\inj_{b_{1}},T\inj_{b_{2}} \rcotuple \comp
      T \lcotuple g_{1},g_{2} \rcotuple \\ 
      & = & 
      \beta_{1} \push{a} \beta_{2} \comp 
      \lcotuple Tg_{1},Tg_{2} \rcotuple \\
      & = & \lcotuple g_{1},g_{2} \rcotuple \comp \gamma \\
      & = & k \comp \gamma \,.
    \end{eqnarray*}
    On the other hand, for $i = 1,2$, we let $l_{i} = \inj_{b_{i}}$.
    Then $l_{i}:(b_{i},\beta_{i},g_{i}) \rTo (d,\delta,k)$ is a
    morphism in $\mathcal{A}_{T}/(c,\gamma)$:
    \begin{eqnarray*}
      \inj_{b_{i}} \comp \delta & = & 
      \inj_{b_{i}} \comp (\beta_{1} \push{a} \beta_{2} \comp 
      \lcotuple T\inj_{b_{1}},T\inj_{b_{2}} \rcotuple) \\
      & = & \beta_{i} \comp \inj_{Tb_{i}}\comp 
      \lcotuple T\inj_{b_{1}},T\inj_{b_{2}} \rcotuple \\
      & = & \beta_{i} \comp T\inj_{b_{i}}\,, \\
      \inj_{i} \comp k & = & \inj_{i} \comp \lcotuple g_{1},g_{2} \rcotuple \\
      & = & g_{i}\,.
    \end{eqnarray*}
    Finally $h_{1} \comp l_{1} = h_{2} \comp l_{2}: (a,f) \rTo (d,k)$ is
    a commutative diagram in $\mathcal{A}/c$.
  \end{proof}
  
  
  \begin{prpstn}
    Let $\Cat{C},\Cat{W}$ be locally $\lambda$-presentable categories
    and $T: \Cat{C} \times \Cat{W} \rTo \Cat{C}$ be a
    $\lambda$-accessible functor, $\lambda$ being a regular cardinal
    strictly greater than $\omega$.  Then for each object $w$ of
    $\Cat{W}$ a final $T(-,w)$-coalgebra $(\bd{x}_{w},\struct{x}_{w})$
    exists and the induced functor $\bd{x}:\Cat{W} \rTo \Cat{C}$ is
    again $\lambda$-accessible.
  \end{prpstn}
  \begin{proof}
    The existence of the final coalgebras
    $(\bd{x}_{w},\struct{x}_{w})$ has been proved for example in
    \cite{barr}.  Let $J$ be a $\lambda$-directed poset and $w : J
    \rTo \Cat{W}$ be functor with limiting cocone
    $$
    (\inj_{w_{j}}:w_{j}\rTo w)_{j \in J}\,.
    $$
    The arrows $\struct{x}_{w_{j}}:\bd{x}_{w_{j}} \rTo^{}
    T(\bd{x}_{w_{j}},w_{j})$ are natural in $j$, so that we can take
    their colimit to construct the $T(-, w)$-coalgebra
    $$
    \colim \bd{x}_{w_{j}} \rTo[l>=4em]^{\colim\struct{x}_{w_{j}}}
    \colim T(\bd{x}_{w_{j}} , w_{j}) \rTo[l>=4em]^{\lcotuple
      T(\inj_{\bd{x}_{w_{j}}},\inj_{w_{j}}) \rcotuple} T(\colim
    \bd{x}_{w_{j}} ,w)\,.
    $$
    We claim that this is a final $T(-, w)$-coalgebra. To prove the
    claim it is enough to show that if a the carrier of a $T(-,
    w)$-coalgebra $(a,\alpha)$ is $\lambda$-presentable in $\Cat{C}$,
    then there exists a unique coalgebra morphism into the coalgebra
    defined above.  Indeed, we have shown that $\mathcal{A}_{T(-,w)}$
    is a strong generator for the category of $T(-,w)$-coalgebras, so
    that the claim follows by considering that every object of
    $\Cat{C}_{T(-,w)}$ is a colimit of objects from
    $\mathcal{A}_{T(-,w)}$.
  
    Consider therefore a coalgebra $(a,\alpha)$ in
    $\mathcal{A}_{T(-,w)}$. Then we can factor $\alpha$ as $\alpha_{j}
    \comp T(a,\inj_{w_{j}})$ for some $j \in J$, since
    $(T(a,\inj_{w_{j}}):T(a,w_{j})\rTo T(a,w))_{j \in J}$ is a
    colimiting cone, the functor $T(a, -)$ being $\lambda$-accessible.
    Let $f_{j}$ be such that $f_{j} \comp \struct{x}_{w_{j}} =
    \alpha_{j} \comp T(f_{j},w_{j})$, then $f_{j} \comp
    \inj_{\bd{x}_{w_{j}}}$ is a morphism of coalgebras, as shown in
    the next diagram:
    $$
    \mydiagrambot[7em]{ [] ( !S {a}{T(a,w_{j})}
      {\bd{x}_{w_{j}}}{T(\bd{x}_{w_{j}},w_{j})} {1}{1.4}, !a
      {^{\alpha_{j}}}{^{f_{j}}}
      {^/-1.5mm/{T(f_{j},w_{j})}}{^{\struct{x}_{w_{j}}}} "1"="a"
      "2"="T(a,w_{j})" "3"="\bd{x}_{w_{j}}" ) "T(a,w_{j})" ( !S
      {T(a,w_{j})}{T(a,w)}
      {T(\bd{x}_{w_{j}},w_{j})}{T(\bd{x}_{w_{j}},w)} {1}{1.4}, !A
      {T(a,\inj_{w_{j}})}{}
      {T(f_{j},w)}{T(\bd{x}_{w_{j}},\inj_{w_{j}})} "2"="T(a,w)"
      "4"="T(\bd{x}_{w_{j}},w)" ) "\bd{x}_{w_{j}}"="1" (
      !S{\bd{x}_{w_{j}}}{T(\bd{x}_{w_{j}},w_{j})} {\colim
        \bd{x}_{w_{j}}}{\colim T(\bd{x}_{w_{j}},w_{j})} {1}{1.4} !a
      {}{^{\inj_{\bd{x}_{w_{j}}}}}
      {^/-2mm/{\inj_{T(\bd{x}_{w_{j}},w_{j})}}}{|{\colim
          \struct{x}_{w_{j}}}} ) "2" (
      !S{T(\bd{x}_{w_{j}},w_{j})}{T(\bd{x}_{w_{j}},w)} {\colim
        T(\bd{x}_{w_{j}},w_{j})} {T(\colim \bd{x}_{w_{j}},w)} {1}{1.4}
      !a{}{} {^/-0.9em/{T(\inj_{\bd{x}_{w_{j}}},w)}} {|{\lcotuple
          T(\inj_{\bd{x}_{w_{j}}},\inj_{w_{j}}) \rcotuple}} )
      "a":@/_3em/"T(a,w)"_{\alpha} }
    $$
    
    On the other hand, suppose that $\alpha \comp T(f,w) = f \comp
    \colim \struct{x}_{w_{j}} \comp \lcotuple
    T(\inj_{\bd{x}_{w_{j}}},\inj_{w_{j}}) \rcotuple$ and choose
    liftings $\alpha_{i}$ and $f_{j}$ of $\alpha$ and $f$
    respectively, as usual, since $a$ is $\lambda$-presentable.
    Moreover, since $J$ is directed, we can assume that $i = j$.
  
    Recall that
    $$
    (T(\inj_{\bd{x}_{w_{j}}},\inj_{w_{j}}): T(\bd{x}_{w_{j}},w_{j})
    \rTo[l>=4em]^{} T(\colim \bd{x}_{w_{j}}, w))_{j \in J}
    $$
    is a colimiting cocone, hence $f_{j} \comp \struct{x}_{w_{j}}$
    and $\alpha_{j} \comp T(f_{j},w_{j})$ are two different liftings of
    the same arrow to this colimit:
    \begin{eqnarray*}
      \alpha \comp T(f,w) 
      & = & \alpha_{j} \comp T(a, \inj_{w_{j}}) \comp
      T(f_{j},w) \comp T(\inj_{\bd{x}_{w_{j}}},w)  \\
      & = &
      \alpha_{j} \comp T(f_{j},w_{j}) \comp
      T(\inj_{\bd{x}_{w_{j}}},\inj_{w_{j}}) \\[2mm]
      \alpha \comp T(f,w) & = & f \comp \colim
      \struct{x}_{w_{j}} \comp \lcotuple T(\inj_{\bd{x}_{w_{j}}},\inj_{w_{j}})\rcotuple
      \\
      & = & f_{j}\comp \inj_{\bd{x}_{w_{j}}}\comp \colim
      \struct{x}_{w_{j}} \comp \lcotuple T(\inj_{\bd{x}_{w_{j}}},\inj_{w_{j}}) \rcotuple
      \\
      & = & f_{j}\comp
      \struct{x}_{w_{j}} \comp  T(\inj_{\bd{x}_{w_{j}}},\inj_{w_{j}})\,.
    \end{eqnarray*}
  
    Therefore there exists a $k \in J$ such that $j \leq k$ and
    moreover such that
    \begin{eqnarray*}
      f_{j} \comp \struct{x}_{w_{j}} \comp T(\bd{x}_{w_{jk}},w_{jk})
      & = & \alpha_{j} \comp T(f_{j},w_{j})\comp T(\bd{x}_{w_{jk}},w_{jk})\,.
    \end{eqnarray*}
    By naturality of $\struct{x}$, we have
    \begin{eqnarray*}
      \struct{x}_{w_{j}} \comp T(\bd{x}_{w_{jk}},w_{jk})
      & = & \bd{x}_{w_{jk}} \comp \struct{x}_{w_{k}}\,,
    \end{eqnarray*}
    and therefore we obtain the relation
    \begin{eqnarray*}
      f_{j} \comp \bd{x}_{w_{jk}} \comp \struct{x}_{w_{k}}
      & = & \alpha_{j} \comp T(a,w_{jk}) \comp T(f_{j}\comp \bd{x}_{w_{jk}},w_{k})\,.
    \end{eqnarray*}
    We conclude that $f_{j} \comp \bd{x}_{w_{jk}}$ is uniquely
    determined as the unique $T(-,w_{k})$-coalgebra morphism from
    $\alpha_{j} \comp T(a,w_{jk})$ to the final one, and consequently
    also $f$ is uniquely determined, because of
    \begin{eqnarray*}
      f & = & f_{j} \comp \inj_{\bd{x}_{w_{j}}} \\
      & = & f_{j} \comp \bd{x}_{w_{jk}}\comp \inj_{\bd{x}_{w_{k}}}\,.
    \end{eqnarray*}
    In order to conclude the argument it is enough to show $\lcotuple
    \bd{x}_{\inj_{w_{j}}} \rcotuple$ is a morphism of $T(-,w)$-algebras
    between two initial ones:
    \begin{eqnarray*}
      \lefteqn[2cm]{\inj_{\bd{x}_{w_{i}}} \comp\lcotuple
        \bd{x}_{\inj_{w_{j}}} \rcotuple \comp \struct{x}_{w}} \\
      & = & \bd{x}_{\inj_{w_{j}}} \comp \struct{x}_{w} \\
      & = & \struct{x}_{w_{j}} \comp  
      T(\bd{x}_{\inj_{w_{j}}},\inj_{w_{j}})  \\
      & = & \struct{x}_{w_{j}} \comp  
      T(\inj_{\bd{x}_{w_{i}}} \comp \lcotuple
      \bd{x}_{\inj_{w_{j}}} \rcotuple,\inj_{w_{j}})  \\
      & = &  \struct{x}_{w_{j}} \comp T(\inj_{\bd{x}_{w_{i}}},\inj_{w_{i}}) 
      \comp T(\lcotuple
      \bd{x}_{\inj_{w_{j}}} \rcotuple,w) \\
      & = &  \struct{x}_{w_{j}} \comp \inj_{T(\bd{x}_{w_{i}},w_{i})} 
      \comp \lcotuple T(\inj_{\bd{x}_{w_{j}}},\inj_{w_{j}}) \rcotuple 
      \comp T(\lcotuple
      \bd{x}_{\inj_{w_{j}}} \rcotuple,w) \\
      & = &  \inj_{\bd{x}_{w_{i}}} \comp
      \colim \struct{x}_{w_{j}}  
      \comp  \lcotuple T(\inj_{\bd{x}_{w_{j}}},\inj_{w_{j}}) \rcotuple \comp T(\lcotuple
      \bd{x}_{\inj_{w_{j}}} \rcotuple,w) \,,
    \end{eqnarray*}
    so that $\lcotuple \bd{x}_{\inj_{w_{j}}} \rcotuple$ is invertible.
  \end{proof}
\end{labripreprint}

