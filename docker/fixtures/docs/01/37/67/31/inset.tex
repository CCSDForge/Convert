We recall the game-theoretic interpretation of a parity game $G =
\langle S,\hg,\kappa, \epsilon \rangle$, as a two person game $G(E)$,
parameterized in a choice of sets $E = \{ E_{x} \}_{x \in
  P_{\omega}}$.  The graph $S = \langle P,M,\dom,\cod \rangle$ is a
board with a set of positions $P$ and a set of allowed moves $M$. A
move $m \in M$ is from position $\dom(m)$ to position $\cod(m)$; the
graph $S$ has multiple edges, hence distinct moves relating the same
pair of positions are allowed.  From a position $p$, the set of moves
$M_{p} = \dom^{-1}(p)$ is available, and player $\epsilon(p)$ among
players $\sigma$ and $\pi$ must choose how to move.  If he cannot
move, then he loses.  An infinite play $\gamma_{0} \rightarrow
\gamma_{1}\rightarrow \ldots \gamma_{n} \rightarrow \ldots $ is a win
for player $\sigma$ if and only if $\kappa(\, \max \In \gamma\,) =
\nu$, the set $\In \gamma$ being defined by equation (\ref{eq:in}).
If a play ends in a position $x \in P_{\omega}$, then player $\sigma$
must choose an element $e \in E_{x}$ and then he wins; if $E_{x} =
\emptyset$, then he loses.

A typical element of an inductively defined set is a kind of finite
tree; on the other hand, a typical element of a coinductively defined
set is a kind of infinite tree. We shall see that a similar tree-like
representation is available for parity functors on the category of
sets.

\begin{dfntn}
  Let $\langle S,s_{0} \rangle$ be a pointed graph, a \emph{tree} $T$
  over $\langle S,s_{0} \rangle$ is a non-empty collection of finite
  paths $\gamma$ in $S$ such that $\dom \gamma = s_{0}$, which is
  moreover closed under prefixes: if $\gamma_{1}
  \compp \gamma_{2} \in T$, then $\gamma_{1} \in T$.
\end{dfntn}

Observe that a tree $T$ over $\langle S, s_{0} \rangle$ is itself a
graph if we set $\gamma \rightarrow \gamma'$ if and only if $\gamma' =
\gamma \compp m$ for some $m \in M_{\cod \gamma}$; moreover $\cod: T
\rTo S$ is a morphism of graphs. In particular it makes sense to talk
about an infinite path in $T$.
\begin{dfntn}
  Let $G$ be a parity game and let $E = \{ E_{x} \}_{x \in
    P_{\omega}}$ be a collection of sets. A \emph{deterministic winning
  strategy} for player $\sigma$ from position $p \in P$ in the
  game $G(E)$ is a pair $\langle T, \lambda \rangle$ where $T$ is a
  tree over $\langle S,p \rangle$ with the following properties:
  \begin{itemize}
  \item If $\gamma \in T$, $\epsilon( \cod\gamma ) = \pi$ and $m \in
    M_{\cod \gamma}$, then $\gamma \compp m \in T$.
  \item If $\gamma \in T$ and $\epsilon( \cod\gamma )= \sigma$, then
    there exists a unique $m \in M_{\cod \gamma}$ such that $\gamma
    \compp m \in T$.
  \item Every infinite path in the tree $T$ is a win for player
    $\sigma$, that is, if $\gamma: \hat{\omega} \rTo T$, then
    $\kappa(\, \max \In (\gamma \comp \cod)\,) = \nu$.
  \end{itemize}
  On the other hand, $\lambda$ is a labeling of paths $\gamma \in T$
  such that $\cod \gamma \in P_{\omega}$ by an element $e =
  \lambda(\gamma) \in E_{\cod \gamma}$.  We let $\mathcal{S}_{G,p}(E)$
  be the set of deterministic winning strategies for player $\sigma$
  in the game $G(E)$ from position $p$.
\end{dfntn}

We shall often use $\tree{S},\lab{S}$ for the tree and the label of a
strategy $S$, so that $S = \langle\tree{S},\lab{S} \rangle$.  If
$\langle T,\lambda\rangle, \langle R,\rho\rangle \in
\mathcal{S}_{G,p}(E)$, then we shall write $\langle T,\lambda\rangle
\subseteq \langle R,\rho\rangle$ to mean that $T \subseteq R$ and
$\lambda(\gamma) = \rho(\gamma)$ for all $\gamma \in T$ such that
$\cod \gamma \in P_{ \omega}$.
\begin{lmm}
  \label{lem:subset}
  If $\langle T,\lambda\rangle \subseteq \langle
  R,\rho\rangle$, then $T = R$ and $\lambda =
  \rho$.
\end{lmm}
\begin{proof}
  By induction on the length of $\gamma \in R$. If $\length{\gamma} =
  0$, then $\gamma = 1_{p}$. Since $T$ is non empty, $1_{p} \compp
  \gamma' \in T$, hence $1_{p} \in T$ ($1_{p}$ belongs always to a
  winning strategy from position $p$).  If $\length{\gamma} = n + 1$,
  then we can write $\gamma = \gamma' \compp m$ where
  $\length{\gamma'} = n$.  Since $\gamma' \in R$, by the induction
  hypothesis $\gamma' \in T$ as well. If $\epsilon(\cod \gamma' ) =
  \pi$, then $\gamma' \compp m' \in T$ for each $m' \in M_{\cod
    \gamma'}$, in particular $\gamma = \gamma' \compp m \in T$.  If
  $\epsilon(\cod \gamma') = \sigma$, then there exists $m' \in M_{\cod
    \gamma'}$ such that $\gamma' \compp m' \in T$.  This implies that
  $\gamma' \compp m',\gamma' \compp m \in R$ and $m = m'$, since $R$
  is deterministic. Therefore $\gamma' \compp m \in T$.  Finally, let
  $\gamma \in R$ be such that $\cod \gamma \in P_{\omega}$.  Since we
  have seen that $\gamma \in T$, it follows that $\rho(\gamma) =
  \lambda(\gamma)$.
\end{proof}

Observe that $\mathcal{S}_{G,p}$ is a functor from the category
$\Sets^{P_{\omega}}$ to $\Sets$, the category of sets and functions.
Given a collection of functions $\{ f_{x} : E_{x} \rTo F_{x} \}_{x \in
  P_{\omega}}$, we can transform a strategy $\langle T, \lambda
\rangle \in \mathcal{S}_{G,p}(E)$ into the strategy $\langle T,
f_{\cod}\circ \lambda\rangle \in \mathcal{S}_{G,p}(F)$, where
\begin{eqnarray*}
  (\,f_{\cod}\circ \lambda\,)(\,\gamma\,) & = & f_{\cod \gamma}(\,\lambda(\gamma)\,)\,.
\end{eqnarray*}
Thus, we denote by $\mathcal{S}_{G} : \Sets^{P_{\omega}} \rTo
\Sets^{P_{\leq n}}$ the functor whose $p$-projection is
$\mathcal{S}_{G,p}$, i.e. $\mathcal{S}_{G}(E) = \ltuple
\mathcal{S}_{G,p}(E) \rtuple_{p \in P_{\leq n}}$.  The following
theorem is the main result of this section.
\begin{thrm}
  \label{theo:mainresult}\label{prop:strategies}
  The equality
  \begin{eqnarray*}
    \val{G}(E) & := & 
    \mathcal{S}_{G}(E)
  \end{eqnarray*}
  holds.
\end{thrm}
The equality above means that the collection of sets of deterministic
winning strategies satisfies the universal property involved in the
definition of the parity functor, so that it can be taken to be a
concrete representation of the functor. This equality is reminiscent
of the formula of the Propositional Modal $\mu$-Calculus which
describes the set of winning position for player $\sigma$ in a parity
game, cf. \cite{emersonjutla,BRICS-RS-96-54}.

In the rest of this section we prove theorem \ref{theo:mainresult}.
This is done by induction on the height, observing that it holds in an
obvious way if the height of $G$ is $0$.  Thus we shall suppose in the
following that $\height{G} = n > 0$ and that the statement holds for
the predecessor game $P(G)$. To develop the proof, we shall need a
modified version of the predecessor game that does not completely
forget about the structure in the region of maximal finite height. The
delooping game, displayed on the left in figure \ref{fig:D(L)}, is
devised to detect first passages through the region of maximal finite
height in a parity game.
\begin{figure}[h]
  \centering
  \vspace{-3mm}
  $$    
  \mygame[3em]{
    []
    ([d]!O{30}([r]!P{40})
    [d]!P{50}([r]!P{60})
    [d]!P{70}([r]!O{80})
    )
    [r(3)]!E{11}([r]!E{21}="A")
    [d]!E{31}([r]!E{41})
    [d]!E{51}([r]!E{61})
    [d]!E{71}([r]!E{81})
    "30"(:@/^0.5em/"11",:@/_0.5em/"51")
    "40"(:@/_0.0em/"31",:@/_0.1em/"61")
    "60"(:"80",:@/_0.4em/"41")
    "80"(:"70",:@`{"61"+(1,-1),"41"+(0.5,-0.5)}"21"))
    "50"(:"70",:"60")
    "70"(:"60")
    "11"("81"!L{}{\omega})
    "30"("40"!L{\mu}{3})
    "50"("60"!L{\nu}{2})
    "70"("80"!L{\mu}{1})
  }
  \hspace{-5mm}\rTo[l>=4em]^{\Phi}\hspace{-5mm}
  \mygame[3em]{ 
    []!E{1}([r]!E{2}="A") [d]!O{3}([r]!P{4})
    [d]!P{5}([r]!P{6}) [d]!P{7}([r]!O{8}) "3"(:"1",:"5")
    "5"(:"7",:"6") "7"(:"6") "4"(:"3",:@/^0.2em/"6")
    "6"(:"8",:@/^0.7em/"4") "8"(:"7",:@/_2em/"2") "1"("2"!L{}{\omega})
    "3"("4"!L{\mu}{3}) "5"("6"!L{\nu}{2}) "7"("8"!L{\mu}{1}) 
  }
  $$
  \vspace{-12mm}
  \caption{The delooping game of a parity game.}
  \label{fig:D(L)}
\end{figure}

\begin{dfntn}
  The \emph{delooping game} $D(G) = \langle S',\hg',\kappa',\epsilon'
  \rangle$ of a parity game $G$ is defined as follows:
  \begin{itemize}
  \item $S'$ is the graph whose set of positions is $P_{\leq
      n}\times\{0\} + P\times\{1\}$. For each move $m: p \rightarrow
    p'$ there is a move $(m,0): (p ,0) \rightarrow (p' ,i)$, where $i
    = 1$ if and only if $p' = \cod m \in P_{n} \cup P_{\omega}$ or $p
    = \dom m \in P_{n}$.
  \item $\hg'(p,0) = \hg(p)$ and $\hg(p,1) = \omega$.
  \item $\kappa'(i) = \kappa(i)$ for $i = 1,\ldots ,n = \height{D(G)}=
    \height{G}$.
  \item $\epsilon'(p,0) = \epsilon(p)$.
  \end{itemize}
  The delooping game $D(G)$ comes with a morphism of graphs $\Phi : S'
  \rTo S$, defined by $\Phi(p,i)= p$ and $\Phi(m,0) = m$.
\end{dfntn}


The domain of the functor $\mathcal{S}_{D(G)}$ is the category
$\Sets^{P_{\leq n}\times \{1 \}} \times \Sets^{P_{\omega}\times
  \{1\}}$ and its codomain is the category $\Sets^{P_{\leq n}\times \{
  0 \}}$. Under the obvious isomorphisms $P \iso P \times \{1 \}$ and
$ P_{\leq n} \iso P_{\leq n} \times \{ 0 \}$ we shall look at this
functor as having the shape
$$
\mathcal{S}_{D(G)} : \Sets^{P_{\leq n}} \times \Sets^{P_{\omega}}
\rTo \Sets^{P_{\leq n}}\,.
$$
\begin{lmm}
  There is a natural isomorphism
  \begin{eqnarray*}
    \mathcal{S}_{D(G)}
    &\iso & \langle 
    \mathcal{S}_{P(G)} \circ \prj_{P_{n} \cup P_{\omega}},
    \EQ_{G} \rangle\,.
  \end{eqnarray*}
\end{lmm}
\begin{proof}
  Let $\{ E_{x} \}_{x \in P}$ be a collection of sets.  Observe that
  from a position $(p,0)$ with $p \in P_{< n}$ the game $D(G)(E)$ is
  exactly as the game $P(G)(E')$, $E'$ being the collection $\{ E_{x}
  \}_{x \in P_{n} \cup P_{\omega}}$.  On the other hand, if $p \in
  P_{n}$ then a strategy from position $(p,0)$ in $D(G)(E)$ is given by
  the choice of a tuple $\{ e_{m} \in E_{\cod m}\}_{m \in M_{p}}$
  if $\epsilon(p) = \pi$, or by the choice of a pair $(e,m)$ with $m
  \in M_{p}$ and $e \in E_{\cod m}$ if $\epsilon(p ) = \sigma$.  
\end{proof}

Therefore, by the previous lemma and by the induction hypothesis
$\val{P(G)}
\linebreak
\iso \mathcal{S}_{P(G)}$, in order to prove theorem
\ref{theo:mainresult} it is enough to show that the collection of sets
$\mathcal{S}_{G}(E)$ carries an invertible algebra structure for the
functor $\mathcal{S}_{D(G)}(-,E)$, and that this structure leads to an
initial algebra if $\kappa(n) = \mu$ and to a final coalgebra if
$\kappa(m) = \nu$.

Observe that an infinite play $\delta$ is a win for player $\sigma$
in $D(G)$ if and only if $\Phi \delta$ is an infinite winning play for
player $\sigma$ in $G$. Then, it is informally seen that a winning
strategy $\langle R,\rho \rangle$ for player $\sigma$ from a position
$(p,0)$ in $D(G)(\mathcal{S}_{G}(E),E)$ gives rise to a strategy from
position $p$ in $G(E)$ as follows: player $\sigma$ uses $\langle R,
\rho \rangle$ as far as he can, by identifying a position $p$ to the
position $(p,0)$; if this is not anymore possible, since a position of
the form $(p',1)$ with $p' \in P_{\leq n}$ has been reached by means of
a play $\delta \in R$, then player $\sigma$ continues according to the
strategy $\rho(\delta) \in \mathcal{S}_{G,p'}(E)$. Moreover, every
winning strategy from $p$ in $G(E)$ arises in this way. We formalize
these ideas next.


\begin{dfntn} \hspace{2mm}
  \begin{itemize}
  \item A path $\delta$ of $D(G)$ is called an \emph{atom} if
    $\length{\delta} > 0$, $\cod \delta \in P'_{\omega}$ and $\cod
    \Phi\delta \in P_{\leq n}$. If $R$ is a set of paths in $D(G)$,
    then we shall write $A(R)$ for the set of atoms of $R$.
  \item Let $\delta$ be a path of $G$ and let $T$ be a collection of
    paths of $G$ with domain $\cod \delta$. By $\delta \compp T$ we
    mean the set $\{ \,\delta \compp \gamma \, |\, \gamma \in T \,\}$.
  \item Let $\langle T ,\lambda \rangle \in \mathcal{S}_{G,p}(E) $ and
    $\delta \in T$.  By $\delta \res \langle T ,\lambda \rangle$ we
    denote the winning strategy $\langle T', \lambda' \rangle \in
    \mathcal{S}_{G, \cod \delta}(E)$ where $T'$ is the set $\{
    \,\gamma\,| \,\delta \compp \gamma \in T \,\}$ and
    $\lambda'(\gamma) = \lambda( \delta \compp \gamma )$ if $\gamma
    \in T'$ and $\cod \gamma \in P_{\omega}$. We call this strategy
    \emph{the residual strategy} of $\langle T,\lambda\rangle$ after
    the path $\delta$.
\end{itemize}
\end{dfntn}

\begin{lmm}
  The pair $\langle T' ,\lambda' \rangle = \delta \res \langle
  T,\lambda\rangle$ is a winning strategy for player $\sigma$ in
  the game $G(E)$ from position $\cod \delta$.
\end{lmm}
\begin{proof}
  Let $\gamma \in T'$. If $\cod \gamma \in P_{\leq n}$, then
  $\cod(\delta \compp \gamma) = \cod \gamma \in P_{\leq n}$ as
  well.  If $\epsilon(\cod \gamma) = \pi$, then $\delta \compp
  \gamma \compp m \in T$ and $\gamma \compp m \in T'$ for all $m
  \in M_{\cod \gamma}$. If $\epsilon(\cod \gamma) = \sigma$, then
  there exists an $m \in M_{\cod \gamma}$ such that $\delta \compp
  \gamma \compp m \in T$, and henceforth $\gamma \compp m \in T'$.
  If $\gamma \compp m' \in T'$, then $\delta \compp \gamma \compp
  m' \in T$ and therefore $m = m'$.  Let $\gamma$ be an infinite
  path in $T'$, then $\delta \compp \gamma$ is an infinite path in
  $T$ and since $ \In( \gamma ) = \In( \delta \compp\gamma )$ we
  see that
  \begin{eqnarray*}
    \kappa(\max \In( \gamma )) 
    & = & \kappa(\max\In( \delta \compp\gamma )) \\
    & = & \nu\,.
  \end{eqnarray*}
\end{proof}

\begin{dfntn}
  We define an algebra
  $$
  \struct{x} : 
  \mathcal{S}_{D(G)}(\mathcal{S}_{G}(E),E)
  \rTo 
  \mathcal{S}_{G}(E)
  $$
  in the following way. If $\langle R, \rho \rangle \in
  \mathcal{S}_{D(G)}(\mathcal{S}_{G}(E),E)$, then we let
  $\struct{x}_{p}(R,\rho) = \langle \low{R},\low{\rho} \rangle$ where
  \begin{eqnarray*}
    \low{R} & = & 
    \Phi R \cup 
    \bigcup_{\delta \in A(R)} 
    \Phi\delta \compp \tree{\rho(\delta)}\,, \\
    \low{\rho}(\gamma)
    & = & \left \{ 
      \begin{array}{ll}
        \rho(\delta), & \gamma = \Phi\delta,\, \delta\in R, \\
        \lab{\rho(\delta)}(\gamma'), 
        & \gamma = (\Phi\delta) 
        \compp \gamma', \,\delta \in A(R),
      \end{array}
    \right .
  \end{eqnarray*}
  if $\gamma \in \low{R}$ and $\cod \gamma \in P_{\omega}$.
\end{dfntn}
\begin{lmm}
  $\chi_{p}( R, \rho )$ is a winning strategy for player $\sigma$
  in the game $G(E)$ from position $p$.
\end{lmm}
\begin{proof}
  Let $\gamma \in \low{R}$ be such that $\cod \gamma \in P_{\leq
    n}$. Either $\gamma =\Phi\delta$ where $\delta \in R$ is not
  an atom, or $\gamma = \Phi\delta \compp \gamma'$ where $\delta
  \in A(R)$ and $\gamma' \in \tree{\rho(\delta)}$.  Similarly, an
  infinite path $\gamma$ in $\low{R}$ is either the image of
  infinite path in $R$ or it is of the form $\gamma = \Phi\delta
  \compp \gamma'$ where $\delta \in A(R)$ and $\gamma'$ is an
  infinite path in $\tree{\rho(\delta)}$. The desired properties of
  $\low{R}$ follow then from the properties of $R$ and from the
  properties of $\rho(\delta)$, respectively. For example, let
  $\gamma =\Phi\delta$ where $\delta \in R$ is not an atom.
  Suppose that $\epsilon(\cod \Phi\delta) = \sigma$ and observe
  that $\epsilon(\cod \delta) = \sigma$ as well, since $\delta$ is
  not an atom. Hence we can find $m \in M_{\cod \delta}$ such that
  $\delta \compp m \in R$ and therefore $\Phi\delta \compp \Phi m
  \in \low{R}$. If $\Phi\delta \compp m' \in \low{R}$, then we can
  find $\delta'$ such that $\Phi \delta' = \Phi\delta \compp m'$.
  It follows that $\delta' = \delta \compp m''$, where $\Phi m'' =
  m'$. Since $R$ is deterministic, $m'' = m$ and hence $m' = \Phi m$.
\end{proof}   
 
\newpage
\begin{dfntn}
  We define a coalgebra
  $$
  \struct{y} : 
  \mathcal{S}_{G}(E)\rTo
  \mathcal{S}_{D(G)}(\mathcal{S}_{G}(E),E)
  $$
  in the following way. If $\langle T, \lambda \rangle
  \in\mathcal{S}_{G}(E)$, then we let $\struct{y}_{p}(T,\lambda)
  = \langle \rise{T},\rise{\lambda} \rangle$ where
  \begin{eqnarray*}
    \rise{T} & = & \{ \, \delta \,|\, \Phi\delta \in T\, \} \\
    \rise{\lambda}(\delta) & = & 
    \left \{
      \begin{array}{ll}
        \lambda(\Phi\delta),& \cod \Phi\delta \in P_{\omega}, \\
        \Phi\delta \res \langle T,\lambda\rangle, 
        & \cod \Phi\delta \in P_{\leq n},
      \end{array}
    \right .
  \end{eqnarray*}
  if $\delta \in \rise{T}$ and $\cod \delta \in P'_{\omega}$.
\end{dfntn}

\begin{lmm}
  $\struct{y}_{p}(T,\lambda)$ is a winning strategy for player
  $\sigma$ from position $p$ in the game $D(G)(\mathcal{S}_{G}(E),E)$.
\end{lmm}
\begin{proof}
  Let $\delta \in \rise{T}$ and suppose that $m \in M_{\cod \delta}$
  and $\epsilon(\cod \delta) = \pi$. Then $\epsilon(\cod \Phi \delta)
  = \pi$, so that $\Phi\delta \compp \Phi m \in T$ and thus $\delta
  \compp m \in \rise{T}$. If $\epsilon(\cod \delta) = \sigma$, then
  $\delta$ is not an atom.  Since $\epsilon(\cod \Phi \delta) =
  \sigma$, there exists an $m$ such that $\Phi \delta \compp m \in T$.
  Since $\delta$ is not an atom, the move $m$ can be lifted to a move
  $m'\in M_{\cod \delta}$ such that $\Phi m' = m$. It follows that
  $\delta \compp m' \in \rise{T}$.  If $\delta \compp m'' \in
  \rise{T}$, then $\Phi m'' = m$, since $T$ is deterministic, and
  therefore $m'' = m'$ since a lifting of $m$ is unique. Finally, an
  infinite path $\delta$ in $\rise{T}$ gives rise to an infinite path
  $\Phi \delta$ in $T$, which is a win for player $\sigma$ in the game
  $G$.  From this it follows that $\delta$ is win for player $\sigma$
  in $D(G)$.
\end{proof}


\begin{prpstn}
  The functions $\struct{x}_{p}$ and $\struct{y}_{p}$ are
  inverse to each other.
\end{prpstn}
\begin{proof}
  Let $\langle R,\rho\rangle \in
  \mathcal{S}_{D(G),p}(\mathcal{S}_{G}(E),E)$ and $\langle
  T,\lambda \rangle \in \mathcal{S}_{G,p}(E)$.  We shall
  show that
  \begin{eqnarray*}
    \langle R,\rho\rangle \subseteq 
    \struct{y}_{p}(T,\lambda)
    & \tiff &
    \struct{x}_{p}(R,\rho)
    \subseteq \langle T,\lambda\rangle\,.
  \end{eqnarray*}
  The desired result will follow from lemma \ref{lem:subset}.  
  
  Suppose first that $\langle R,\rho\rangle \subseteq
  \struct{y}_{p}(T,\lambda)$. Thus $R \subseteq \Phi^{-1}T$ and $\Phi R
  \subseteq T$, and if $\delta \in R$ is an atom, then
  $\rho(\delta) = \rise{\lambda}(\delta) = \Phi\delta \res \langle
  T,\lambda \rangle$. Hence
  \begin{eqnarray*}
    \low{R} & = & 
    \Phi R \cup \bigcup_{\delta \in A(R)}
    \Phi\delta \compp 
    \tree{\Phi\delta \res \langle T, \lambda \rangle } \\
    & \subseteq & T\,.
  \end{eqnarray*}
  Consider a path $\gamma \in \low{R}$ such that $\cod \gamma \in
  P_{\omega}$. If $\gamma = \Phi\delta$, then $\low{\rho}(\gamma) =
  \rho(\delta) = \rise{\lambda}(\delta) = \lambda(\Phi\delta)$, 
  and if $\gamma =
  \Phi\delta \compp \gamma'$, then $\low{\rho}(\gamma) =
  \lab{\rho(\delta)}(\gamma') =  \lab{\Phi\delta \res \langle
  T,\lambda \rangle}(\gamma') = \lambda(\Phi\delta\compp
  \gamma') = \lambda(\gamma)$.
  
  Suppose now that $\struct{x}_{p}(R,\rho) \subseteq \langle
  T,\lambda\rangle$. Then $\Phi R \subseteq T$ and therefore $R
  \subseteq \Phi^{-1}R = \rise{T}$. Consider a path $\delta \in R$
  such that $\cod \delta \in P'_{\omega}$. If $\cod \Phi\delta \in
  P_{\omega}$ then $\rise{\lambda}(\delta) = \lambda(\Phi\delta) =
  \low{\rho}(\Phi\delta) = \rho(\delta)$; on the other hand, if $\cod
  \Phi\delta \in P_{\leq n}$, then $\rise{\lambda}(\delta) =
  \Phi\delta \res \langle T,\lambda\rangle = \rho(\delta)$, since
  $\rho(\delta) \subseteq \Phi\delta \res \langle T, \lambda \rangle$.
  This can be seen as follows: if $\gamma \in \tree{\rho(\delta)}$ then
  $\Phi\delta \compp \gamma \in \low{R} \subseteq T$, so that $ \gamma
  \in \tree{\Phi\delta \res \langle T,\lambda\rangle}$; if moreover
  $\cod \gamma \in P_{\omega}$, then $\lab{\rho(\delta)}(\gamma) =
  \low{\rho}(\Phi\delta \compp \gamma) = \lambda(\Phi\delta \compp
  \gamma) = \lab{\Phi\delta \res \langle T,\lambda\rangle}(\gamma)$.
\end{proof}


   
\begin{prpstn} 
  \label{prop:initialalgebra}
  If  $\kappa(n) = \mu$, then
  $$
  \struct{x} : 
  \mathcal{S}_{D(G)}(\mathcal{S}_{G}(E),E) \rTo \mathcal{S}_{G}(E)
  $$
  is an initial $\mathcal{S}_{D(G)}(-,E)$-algebra.
\end{prpstn}
\begin{proof}
  First we construct a graph $\mathcal{G}$ as follows: its vertices
  are pairs $(S,p)$ with $p \in P_{\leq n}$ and $S \in
  \mathcal{S}_{G,p}(E)$. A transition of this graph is of the form
  $$
  \delta : (S,p) \rightarrow (\Phi\delta \res S,p')
  $$
  where $\delta$ is an atom such that $\Phi\delta \in S$, $\dom
  \Phi\delta = p$ and $\cod \Phi\delta = p'$. Observe that the graph
  $\mathcal{G}$ is well founded: given an infinite sequence
  $\delta_{i} : (S_{i-1},p_{i-1}) \rightarrow (S_{i},p_{i})$, $i \geq
  1$, we can construct the infinite path $\Phi\delta_{1} \compp
  \Phi\delta_{2}\compp \ldots $ which belongs to $S_{0}$ and
  contradicts the condition on infinite paths for a winning strategy.
  Observe moreover that if $\langle \rise{T},\rise{\lambda}\rangle =
  \struct{y}_{p}(T,\lambda)$ and $\delta \in A(\rise{T})$, then
  $\rise{\lambda}(\delta) = \Phi\delta \res \langle T,\lambda\rangle$,
  hence $\delta : (\langle T,\lambda\rangle,p) \rightarrow \langle
  \rise{\lambda}(\Phi\delta), \cod \Phi\delta\rangle$.
  
  Thus, if $\beta : \mathcal{S}_{D(G)}(B,E)\rTo B$ is another algebra,
  then we can define $f : \mathcal{S}_{G}(E) \rTo B$, by the formula:
    \begin{eqnarray*}
      f_{p}(T,\lambda)
      & = & 
      \beta_{p}(\,\mathcal{\mathcal{S}}_{D(G),p}(f,E)(\,\struct{y}_{p}(T,\lambda)\,)\,)\ \\
      & = & 
      \beta_{p}(\,\mathcal{\mathcal{S}}_{D(G),p}(f,E)(\,\rise{T},\rise{\lambda}\,)\,)\\
      & = & 
      \beta_{p}(\,\rise{T},\lambda'\,)
    \end{eqnarray*} 
    where 
    \begin{eqnarray*}
      \lambda'(\delta)
      & = & 
      \left \{
        \begin{array}{ll}
          \lambda(\delta) , & \cod \Phi\delta \in P_{\omega} \\
          f_{\cod \Phi \delta}(\rise{\lambda}(\delta)), 
          & \delta \textrm{ an atom}
        \end{array}
      \right .
      \end{eqnarray*}
      using the induction hypothesis (on the well founded graph
      $\mathcal{G}$) that we have previously defined $f_{p'}(S')$ for
      each pair $(S',p')$ such that $(\langle T,\lambda \rangle,p)
      \rightarrow (S',p')$.  This is also the unique way to define $f$
      so that $\struct{x} \comp f = \mathcal{S}_{D(G)}(f,E) \comp
      \beta$.  
  \end{proof}
  
  \begin{prpstn}
    \label{prop:finalcolagebra}
    If  $\kappa(n) = \nu$, then
    $$
    \struct{y} 
    : \mathcal{S}_{G}(E) \rTo \mathcal{S}_{D(G)}(\mathcal{S}_{G}(E),E) 
    $$
    is a final $\mathcal{S}_{D(G)}(-,E)$-coalgebra.
  \end{prpstn}
  \begin{proof}
    Consider a coalgebra 
    $$
    \beta : B \rTo
    \mathcal{S}_{D(G)}(B,E)\,,
    $$
    we first define a graph $\mathcal{G}_{\beta}$ as follows: a
    state of $\mathcal{G}_{\beta}$ is a pair $(b,p)$ such that $p \in
    P_{\leq n}$ and $b \in B_{p}$, and a transition $(b,p) \rightarrow
    (b',p')$ of $\mathcal{G}_{\beta}$ is an atom $\delta \in
    \tree{\beta_{p}(b)}$ such that $\cod \Phi\delta = p'$ and
    $\lab{\beta_{p}(b)}(\delta) = b'$. Observe that in the proof of
    proposition \ref{prop:initialalgebra} the graph $\mathcal{G}$
    coincides with the graph $\mathcal{G}_{\struct{y}}$ defined here.
    
    We now define a collection of functions $\{ f_{p} : B_{p} \rTo
    \mathcal{S}_{G,p}(E) \}_{ p \in P_{\leq n}}$ and then split the
    proof of proposition \ref{prop:finalcolagebra} in a sequence of
    lemmas: in \ref{lem:welldefined} we show that these functions are
    well defined, in \ref{lem:morphism} we show that this defines a
    morphism of coalgebras, and finally in \ref{lem:unique} we show
    that this is the unique such morphism.
    \begin{dfntn}
      For each $p \in P_{\leq n}$ and $b \in B_{p}$ we define
      $f_{p}(b) = \langle T_{b}, \lambda_{b} \rangle \in
      \mathcal{S}_{G,p}(E)$ as follows.  We say that $\gamma \in
      T_{b}$ if and only if $\gamma$ has a factorization of the form
      \begin{eqnarray}
        \label{eq:factorization}
        \gamma  & = & 
        \Phi\delta_{1} \compp \ldots 
        \compp \Phi\delta_{k} \compp \Phi\delta_{k
          + 1}
      \end{eqnarray}
      such that
      \begin{itemize}
      \item[1.] $\Delta = (\delta_{1},\ldots ,\delta_{k})$ is a path in
        $\mathcal{G}_{\beta}$ such that $\dom \Delta = (b,p)$ and
        $\cod \Delta = (b',p')$,
      \item[2.] $\delta_{k + 1} \in \tree{\beta_{p'}(b')}$ is not an
        atom.
      \end{itemize}
      If $\cod \gamma \in P_{\omega}$, then we let
      $\lambda_{b}(\gamma) = \lab{\beta_{p'}(b')}(\delta_{k + 1})$.
    \end{dfntn}
    We remark that a factorization of the form
    (\ref{eq:factorization}), without the additional requirements $1.$
    and $2.$, exists for any path $\gamma$ and is unique. This follows
    from the observation that the set of paths
    $\{\,\Phi\delta\,|\,\textrm{$\delta$ is an atom}\,\}$ does not
    contain comparable elements with respect to the prefix order.
    Using the language of the theory of codes \cite{reutenauer}, this
    set is a prefix code. Recall also that $\delta_{k + 1}$ is not an
    atom if either $\length{\delta_{k + 1}} = 0$ or $\cod \delta_{k +
      1} \in P'_{\omega}$ implies $\cod \Phi\delta_{k + 1} \in
    P_{\omega}$.
  
  The game-theoretic interpretation of the strategy $f_{p}(b)$ is as
  follows.  From position $p$, player $\sigma$ uses the strategy
  $\beta_{p}(b)$ as long as he can.  As soon as the play reaches a
  position $p'$ such that either $p'$ in $P_{n}$ or after one move if
  $p \in P_{n}$, this strategy becomes unavailable.  However, if one
  of these two cases happens, the strategy $\beta_{p}(b)$ gives player
  $\sigma$ the choice of an element $b' = \lab{\beta_{p}(b)}(\delta)$
  in $B_{p'}$ and therefore the choice of a new strategy
  $\beta_{p'}(b')$.  Thus player $\sigma$ iterates this process.
  Iteration of this process is expressed by saying that the residual
  strategy of $f_{p}(b)$ after the image of an atom $\delta$ is the
  strategy $f_{p'}(b')$. This is the content of the next lemma.
  \begin{lmm}
    \label{lemma:iteration}
    Let $\langle R, \rho \rangle = \beta_{p}(b)$ and let $\delta
    \in A(R)$. Then
    \begin{eqnarray*}
      \Phi\delta \res f_{p}(b) & =  & f_{\cod
        \Phi\delta}( \rho(\delta)) \,.
    \end{eqnarray*}
  \end{lmm}
  \begin{proof}
    Let $\gamma \in \tree{f_{\cod \Phi\delta}(\rho(\delta))}$, then we
    can write $\gamma = \Phi\delta_{1} \compp \ldots \compp \Phi
    \delta_{k+1}$ and thus $\Phi\delta \compp \gamma = \Phi\delta
    \compp\Phi\delta_{1} \compp \ldots\compp \Phi \delta_{k+1}$ shows
    that $\Phi\delta \compp \gamma \in \tree{f_{p}(b)}$ and $\gamma
    \in \tree{\Phi\delta \res f_{p}(b)}$. If $\cod \gamma \in
    P_{\omega}$ then $\lab{\Phi\delta \res f_{p}(b)}(\gamma) =
    \lab{f_{p}(b)}(\Phi\delta \compp \gamma) =
    \lab{\beta_{p'}(b')}(\delta_{k + 1})$ and similarly $\lab{f_{\cod
        \Phi\delta}(\rho(\delta))}( \gamma) =
    \lab{\beta_{p'}(b')}(\delta_{k + 1})$. 
  \end{proof}
  \begin{lmm}
    \label{lem:welldefined}
    The pair $f_{p}(b) = \langle T_{b},\lambda_{b} \rangle$ is a
    winning strategy for player $\sigma$ in the game $G(E)$ from
    position $p$.
  \end{lmm}
  \begin{proof}
    Let $\gamma \in T_{b}$ have a factorization $\Phi\delta_{1}\compp
    \ldots \compp \Phi\delta_{k+1}$. Observe that $\delta_{1} \in
    \beta_{p}(b)\in \mathcal{S}_{D(G),p}(B,E)$ implies that $\dom
    \delta_{1} = (p,0)$, hence $\dom \gamma = \dom \Phi \delta_{1} =
    p$.  From the definition it is clear that $T_{b}$ is closed under
    prefixes.
    
    Let $\gamma \in T_{b}$ and suppose first that $\epsilon(\cod
    \gamma) = \pi$. If $m \in M_{\cod \gamma}$, then $(m,0) \in
    M_{\cod \delta_{k + 1}}$, since $\delta_{k + 1}$ is not an atom,
    hence $ \delta'_{k + 1} = \delta_{k+1} \compp m \in
    \beta_{p'}(b')$. If $\delta'_{k + 1}$ is not an atom, then we can
    write
    \begin{eqnarray*}
      \gamma  \compp m& = & 
      \Phi\delta_{1} \compp \ldots \compp \Phi\delta_{k} 
      \compp \Phi\delta'_{k
        + 1}
    \end{eqnarray*}
    and if $\delta'_{k + 1}$ is an atom we can write
    \begin{eqnarray*}
      \gamma  \compp m& = & 
      \Phi\delta_{1} \compp \ldots \compp \Phi\delta_{k} 
      \compp \Phi\delta'_{k
        + 1} \compp \Phi 1_{(\cod \Phi \delta'_{k + 1},0)}\,,
    \end{eqnarray*}
    If we let $p'' = \cod \Phi \delta_{k + 1}'$ and $b'' =
    \lab{\beta_{p'}(b')}(\delta_{k +1}')$, then we observe $1_{
      (\cod \delta'_{k + 1},0)} = 1_{(p'',0)}\in
    \tree{\beta_{p''}(b'')}$.  In both cases we conclude that
    $\gamma \compp m \in T_{b}$.
    
    If $\epsilon(\cod \gamma) = \sigma$, then $\epsilon'(\cod
    \delta_{k + 1}) = \sigma$ so that $\delta_{k + 1} \compp (m,0) \in
    \beta_{p'}(b')$ for a unique $m \in M_{\cod \gamma}$. As before,
    we conclude that $\gamma \compp m \in T_{b}$. On the other hand,
    if $\gamma \compp m' \in T_{b}$, then $\delta_{k + 1} \compp
    (m',0) \in \beta_{p'}(b')$, since such a factorization for
    $\gamma$ is unique. Thus $m = m'$, since $\beta_{p'}(b')$ is
    deterministic.
    
    Consider now an infinite path $\gamma$ in $T_{b}$.  Either this
    infinite path visits the region $P_{n}$ infinitely often, in which
    case it is a win for player $\sigma$, or we can write $\gamma =
    \gamma'\compp \Phi\delta$, where $\delta$ is an infinite play in
    $D(G)(B,E)$, played according to a given winning strategy for this
    game.  This infinite play is a win for player $\sigma$ in
    $D(G)(B,E)$ which implies that $\gamma$ is a win for player
    $\sigma$ in $G(E)$.
  \end{proof}
  
  \begin{lmm}
    \label{lem:morphism}
    The  diagram
    $$
    \mydiagram[6em]{ [](!S{B}
      {\mathcal{S}_{D(G)}(B,E)}{\mathcal{S}_{G}(E)}
      {\mathcal{S}_{D(G)}(\mathcal{S}_{G}(E),E)}{1}{2},
      !a{^{\beta}}{^{f}}{^/-0.6em/{\mathcal{S}_{D(G)}(f,E)}}{^{\struct{y}}}
    }
    $$
    commutes.
  \end{lmm}
  \begin{proof}
    It is enough to show that for all $p \in P_{\leq n}$ and $b \in
    B_{p}$
    \begin{eqnarray*}
      \mathcal{S}_{D(G),p}(f,E)(\, \beta_{p}(b)\,)
      & \subseteq & \struct{y}_{p}(\,f_{p}(b)\,)\,.
    \end{eqnarray*}
    Let $\langle T_{b},\lambda_{b}\rangle = f_{p}(b)$ and $\langle R,
    \rho\rangle = \beta_{p}(b)$. If $\delta \in R$, then $\Phi\delta
    \in T_{b}$ so that $\delta \in \rise{T_{b}}$.  Suppose now that
    $\cod \delta \in P'_{\omega}$. If $\cod \Phi\delta \in
    P_{\omega}$, then $\rise{ \lambda_{b}}(\delta)= \lambda_{b}(\Phi
    \delta) = \rho(\delta)$.  If $\cod \Phi \delta \in P_{\leq n}$,
    that is, if $\delta$ is an atom, then $\rise{\lambda_{b}}(\delta)
    = \Phi\delta \res f_{p}(b) = f_{\cod\Phi \delta}( \rho(\delta))$,
    by lemma \ref{lemma:iteration}.
  \end{proof}
  
  \begin{lmm}
    \label{lem:unique}
    If a collection of functions $g: B \rTo \mathcal{S}_{G}(E)$
    satisfies the relation $g \comp \struct{y} = \beta \comp
    \mathcal{S}_{D(G)}(g,E)$, then $g = f$.
  \end{lmm}
  \begin{proof}
    We will prove that $g_{p}(b) \subseteq f_{p}(b)$ for all $p \in
    P_{\leq n}$ and $b \in B_{p}$. In the following let $\beta_{p}(b)
    = \langle R ,\rho\rangle$ and recall that
    $\mathcal{S}_{D(G)}(g,E)(R,\rho) = \langle R ,\rho' \rangle$ where
    $\rho'(\delta) = \rho(\delta)$ if $\cod \Phi \delta \in
    P_{\omega}$ and $\rho'(\delta) = g_{\cod \Phi \delta}(
    \rho(\delta))$ if $\delta$ is an atom. Thus we have reduced the
    relation $g_{p}(b) = \struct{x}_{p}(\,\mathcal{S}_{D(G),p}(g,E)(
    \beta_{p}(b))\,)$ to the relation $g_{p}(b) =
    \struct{x}_{p}(R,\rho')$.
    
    As a first part, we  prove by induction on the length of
    $\gamma$ the following statement: \emph{for each $p \in P_{\leq
        n}$ and $b \in B_{p}$, if $\gamma \in \tree{g_{p}(b)}$,
      then $\gamma \in \tree{f_{p}(b)}$}.
    
    The statement is trivial if $\length{\gamma} = 0$, since if
    $\gamma \in \tree{g_{p}(b)}$, then $\gamma = 1_{p}$ and $1_{p}$
    belongs to any winning strategy from position $p$.  If
    $\length{\gamma} > 0$, we argue using the equality $g_{p}(b) =
    \struct{x}_{p}(R,\rho')$. If $\gamma = \Phi\delta$, then $\gamma
    \in T_{b}$ by its definition. If $\gamma = \Phi\delta \compp
    \gamma'$, where $\delta$ is an atom of $R$ and $\gamma ' \in
    \tree{\rho'(\delta)} = \tree{g_{\cod \Phi\delta}(\rho(\delta))}$,
    then $\gamma' \in \tree{f_{\cod \delta}(\rho(\delta))}$, since
    $\length{\gamma '}< \length{\gamma}$ and using the induction
    hypothesis. Then it is easily seen that $\gamma = \Phi\delta \compp
    \gamma' \in \tree{f_{p}(b)}$ as well.
    
    We now prove again by induction on the length the following
    statement: \emph{for each $p \in P_{\leq n}$ and $b \in B_{p}$, if
      $\gamma \in g_{p}(b)$ and $\cod \gamma \in P_{\omega}$, then
      $\lab{g_{p}(b)}(\gamma) = \lab{f_{p}(b)}(\gamma)$}.
    
    The statement is again obvious if $\length{\gamma} = 0$, since
    there is no such $\gamma$ with $\dom \gamma \in P_{\leq n}$ and
    $\cod \gamma \in P_{\omega}$.  If
    $\length{\gamma} > 0$, then two cases. Either $\gamma = \Phi\delta$ with $\delta
    \in R$, in which case $\lab{g_{p}(b)}(\gamma) =
    \low{\rho'}(\Phi\delta) = \rho'(\delta) = \rho(\delta) =
    \lambda_{b}(\gamma)$ by the definition of $f$.  Or $\gamma =
    \Phi\delta \compp \gamma'$ where $\delta$ is an atom of $R$ and
    $\gamma' \in \tree{\rho'(\delta)} = \tree{g_{\cod \delta}(
      \rho(\delta))}$.  In this case
    $$
    \renewcommand{\arraystretch}{1.3}
    \begin{array}[b]{rcl@{\hspace{7mm}}l}
      \lab{g_{p}(b)}(\gamma) 
      & = &  \low{\rho'}(\Phi\delta \compp \gamma')
      & g_{p}(b) = \struct{x}_{p}(R,\rho')\\
      & = & \lab{\rho'(\delta)}(\gamma') 
      & \textrm{def. of $\struct{x}_{p}$}\\
      & = &
      \lab{g_{\cod
          \Phi\delta}( \rho(\delta))}(\gamma')
      & \textrm{def. of $\rho'$} \\
      & = & \lab{f_{\cod
          \Phi\delta}(\rho(\delta))}(\gamma')  
      & \textrm{induction hypothesis on $\gamma'$}\\
      & = & \lambda_{b}(\Phi\delta \compp \gamma') 
      & \Phi\delta \res f_{p}(b) 
      = f_{\cod \Phi\delta}(\rho(\delta))\\
      & = & \lambda_{b}(\gamma)\,.
    \end{array}
    \renewcommand{\arraystretch}{1}
    $$
  \end{proof}
  This ends the proof of proposition \ref{prop:finalcolagebra} too.
\end{proof}
  
Thus we have completed the proof of theorem \ref{theo:mainresult}. We
end this section with some examples illustrating the theory so far
developed.

\newcommand{\cons}{\textrm{cons}}
\newcommand{\nil}{\textrm{nil}}

\begin{xmpl}
  We consider the set of finite lists over a set of symbols $E$. This
  is initial algebra of the functor $1 + (Y \times E)$ and therefore
  it is the denotation of the $\mu$-term $\mu_{y}.(\top \vee (y \land
  E))$.
  \begin{figure}[h]
    $$
    \mygame[3em]{
      [d(6)] []!E{\mu} 
      :[d]!E{\vee} 
      ( :[ld]!E{\top}
      [ll]="I1", 
      :[rd]!E{\land} 
      ( :[ld]!E{y}:@`{"I1"}"\mu",
      :[rd]!E{E}
      ) 
      "\land"[r(2)]*+{\leadsto} 
    } 
    \hspace{-0mm} 
    \mygame[3em]{
      []="S"([r]!E{E}) [d(1.5)] [l] :[r]!E{\sigma}="A" (
      :[ld]!E{\pi}|{\makebox[8mm][l]{\nil}} [rr]="I1",
      [rrd]="I2", :[d]!E{\pi}="B"|{\makebox[8mm][l]{\textrm{cons}}} (
      :@`{"I1"}"\sigma", :@`{"I2"}"E" ) "S"
      [l(1.5)u(0.4)]("S"[r(1.5)d(0.2)]!L{}{\omega}) "A"
      [l(1.5)u(0.6)]("B"[r(1.5)d(0.2)]!L{\mu}{1}) "S" [u(7)]
      !E{\sigma} :[d]!E{\pi}|{\makebox[8mm][l]{\cons}}
      (:[r]!E{1}) :[d] !E{\sigma}
      :[d]!E{\pi}|{\makebox[8mm][l]{\cons}} (:[r]!E{0}) :[d]
      !E{\sigma} (:[dl]!E{\pi}|{\makebox[4mm][l]{\nil}})
      [d]:@{{.}{.}{>}}[d] }
    $$\centering\vspace{-8mm}
    \caption{Lists as winning strategies.}
    \label{fig:lists}
  \end{figure}
  In figure \ref{fig:lists} we have translated this $\mu$-term
  into a pointed parity game, according to proposition
  \ref{prop:termstogames} and to a well established practice in the
  model checking community.  The conventions are the ones followed
  until now: positions of the games, labeled by $\sigma$ or $\pi$, are
  grouped within boxes according to their height.  The height is on
  the right of the boxes, the color is on the left. For convenience of
  exposition, we have labeled transitions in the figure, even if this
  is not strictly necessary.
  
  It is immediate to realize that there is a bijection between lists
  and deterministic winning strategies in the parity game. If we let
  $E = \{ 0,1 \}$, we have represented in figure \ref{fig:lists} the
  list $\cons(\cons(\nil,0),1)$ in the form of a winning strategy, the
  tree over the game.  Observe that we cannot obtain infinite lists
  since every infinite path on the corresponding tree would be a loss
  for player $\sigma$.
\end{xmpl}

\begin{xmpl}  
  We want to calculate an algebraic expression describing the set of
  infinite trees with the following properties: 1) every node is
  labeled by an element of a given set $E$, 2) every node has a finite
  (possibly empty) list of sons.  According to experience, this set
  could be expressed as the greatest solution of the equation
  \begin{eqnarray*}
    X & = & E \times X^{*}\,,
  \end{eqnarray*}
  that is, the final coalgebra of the functorial expression on the
  right.  On the other hand, we know that $X^{*}$ is the least
  solution of
  \begin{eqnarray*}
    Y & = & 1 + (Y \times X)\,,
  \end{eqnarray*}
  hence we guess that the desired algebraic expression is given by the
  $\mu$-term $\nu_{x}. (E \land \mu_{y}.  (\top \lor (y \land x ))$.
  We can verify that this guess is correct by transforming the
  $\mu$-term into a pointed parity game, according to proposition
  \ref{prop:termstogames}, the result being the game on the right of
  figure \ref{fig:unica}.
  \begin{figure}[h]
    $$
    \mygame[3em]{
      []!E{\nu}
      :[d]!E{\land}
      (:[ld]!E{E})
      :[rd]!E{\mu}
      :[d]!E{\vee}
      (
      [rrr]="I2",
      :[ld]!E{\top}
      [ll]="I1",,
      :[rd]!E{\land}
      (
      :[ld]!E{y}:@`{"I1"}"\mu",
      :[rd]!E{x}:@`{"I2"}"\nu"
      )
      )
    }
    \hspace{-0.3cm}
    \leadsto
    \hspace{0.0cm}
    \mygame[3em]{
      []
      :[r]!E{\pi}("\pi"="\nu")
      (:@/^2em/[u(1.8)]!E{E}|{\textrm{What label ?}})
      :[d(1.8)]!E{\sigma}="A"|{\makebox[8mm][r]{\textrm{What list is down ?}}}
      ("\sigma"="\mu",
      [rr]="I2", 
      :[ld]!E{\pi}="B"
      |{\makebox[4mm][l]{\nil}}
      ([ddll]="I1",[uull]="II1"), 
      :[rd]!E{\pi}="C"
      |{\makebox[4mm][l]{\cons}}(
      :@`{"I1","II1"}"\mu"|{
        \makebox[15mm][r]{\textrm{What tail ?}}
      }, 
      :@`{"I2"}"\nu"|{\makebox[8mm][l]{\textrm{What subtree ?}}}
      ) 
      )
      "E"
      [l(1.5)u(0.3)]("E"[r(1.5)d(0.2)]!L{}{\omega})
      "\nu"
      [l(1.5)u(0.3)]("\nu"[r(1.5)d(0.2)]!L{\nu}{2})
      "A"
      [l(1.5)u(0.6)]("C"[r(0.5)d(0.2)]!L{\mu}{1})
    }
    $$\vspace{-10mm}
    \caption{Infinite trees as winning strategies.}
    \label{fig:unica}
  \end{figure}
  It is possible to convince ourself that a labeled tree with those
  properties gives rise to a deterministic winning strategy for player
  $\sigma$ by interpreting a move by $\pi$ as a question about the
  tree. Conversely, every such strategy comes from a unique tree of
  this kind.
  
  It is worth examining infinite paths in this game.  Player $\sigma$
  cannot answer that a node has an infinite list of sons: this would
  be done by answering infinitely often ``cons'' to the question
  ``What tail ?'', without being asked the question ``What list is
  down ?''. The region visited infinitely often of maximal height  in
  such a play is colored by $\mu$, hence it is a loss for player
  $\sigma$.  On the other hand, player $\sigma$ can answer infinitely
  often ``cons'' provided the play is going down in examining the
  tree, that is, provided this answer is alternating with the question
  ``What list is down ?''. The maximal region visited infinitely often
  in such a play is colored by $\nu$, hence it is a win for player
  $\sigma$.
\end{xmpl}


\begin{xmpl}
  It is well known that infinite finitely branching trees can be
  encoded as infinite binary trees. Proposition \ref{prop:strategies}
  can be taken to be a generalization of this fact, in that it shows
  that the elements of every nullary parity functor can be encoded as
  infinite
  trees with a bounded out-degree.  
\end{xmpl}

\begin{xmpl}
  Charity \cite{MR96c:18005} is a programming language designed out of
  categorical principles, thus recursion and corecursion are in this
  context synonymous for the universal properties of initial and final
  coalgebras. An important principle of this programming language
  states that it is possible to define an arrow $f:\mu_{x}.T(x) \times
  B \rTo C$ from an algebra in context $g:T(C) \times B \rTo C$,
  provided $T$ is a strong categorical datatype \cite{MR94a:18008}.
  This means that $T$ comes with a natural transformation (a strength)
  $$
  \theta^{T}_{A,B} : T(A) \times B \rTo T(A \times B)
  $$
  satisfying associativity and unitary constrains.  The explicit
  characterization of set-theoretic parity functors allows the direct
  computation of a strength. If $A$ and $B$ are two collections of
  sets indexed by $P_{\omega}$, then we can associate to a strategy
  $\langle T,\lambda\rangle \in \mathcal{S}_{G,p}(A)$ and to a
  collection $b = \{ b_{x} \in B_{x}\}_{x \in P_{\omega}}$ the
  strategy $\langle T,\lambda^{b}\rangle \in \mathcal{S}_{G,p}(A
  \times B)$, where if $\gamma \in T$ and $\cod \gamma \in P_{\omega}$
  then $\lambda^{b}(\cod \gamma) = (\lambda(\cod \gamma),b_{\cod
    \gamma})$.
\end{xmpl}



