Several set-theoretic structures of relevance to computer science can
be described either by using the language of initial algebras or by
using the language of final coalgebras.  For example, sets of finite
trees, the set of terms over a signature and, more in general,
inductively defined sets are initial algebras of some functor and this
property characterizes these sets up to canonical isomorphism.
Similarly, sets of trees with possibly infinite branches, sets of
objects which are canonical solutions of systems of equations,
coinductively defined sets or coinductive types, can be characterized
up to isomorphism by the property of being final coalgebras of some
functor.  Thus the study of initial algebras in connection with data
structures is a well developed subject and dates back at least twenty
five years \cite{MR82d:18004,lehmannsmyth}. The interest in final
coalgebras is more recent, but is nowadays grown up to a well
established discipline \cite{MR89j:03039,barr,MR1791953}.  Despite the
existence of programming languages and proof assistants that implement
both inductive and coinductive types \cite{CoFu92,gimenez}, it appears
to us that initial algebras and final coalgebras are not often studied
simultaneously in the literature. Thus we are led to the problem of
understanding what kind of structures arise from both initial algebras
and final coalgebras and whether these structures are of any interest
in computer science.

This paper is therefore meant to be an introduction to the theory of
categories having the following completeness properties: they have
finite \emph{sums}, finite \emph{products} and all the \emph{initial
  algebras} and \emph{final coalgebras} of the functors which can be
constructed out of these four operations, using projection functors as
building blocks.  These categories generalize $\mu$-lattices
\cite{San99:Fml,BRICS-RS-00-29} and $\mu$-algebras \cite{MR87e:03079}
on one side, on the other side they partially generalize bicomplete
categories \cite{MR96k:18004a}.  For this reason we call them
\emph{$\mu$-bicomplete}.

A first concern is to show that many common categories are
$\mu$-bicomplete. It is well known that an accessible unary
endofunctor of a locally presentable category has both an initial
algebra and a final coalgebra. We adapt ideas existing in the
literature to prove that locally presentable categories are
$\mu$-bicomplete.  Among the locally presentable categories is the
category of sets and functions, which is therefore $\mu$-bicomplete.

 As it is often the case for
existence theorems, the mere knowledge that a category is
$\mu$-bicomplete is unsatisfactory. The principal goal of this paper
is that of giving an explicit description of the functors on the
category of sets that arise out of those four operations, i.e. of the
functors that are definable by $\mu$-terms.   We achieve this goal by translating the algebraic
language of $\mu$-bicomplete categories into the combinatorial
language of \emph{parity games}, cf. \cite[\S 4]{AN01}.  These games
are a standard tool in the theory of automata recognizing infinite
objects \cite{thomas}. A central notion in this theory is that of an
acceptance condition, essentially a method for specifying a set of
infinite paths in a graph. The acceptance condition by which the set
of infinite winning plays in a parity game is defined was introduced
in \cite{MR827531} to construct automata in normal form.  Thus it
should not come as a surprise that several combinatorial problems of
the theory can be reduced to the problem of finding winning strategies
in a parity game.  For example, the properties of transition systems
that are definable by alternating fixed point expressions can be
checked using algorithms designed and proved correct by means of
game-theoretic ideas and analogies \cite{emersonjutla}.

Generalizing ideas that relate the theory of two persons games with
the theory of bicomplete categories \cite{MR96k:18004a,MR1797101}, we
show that it is possible to endow parity games with an algebraic
meaning, so that they can be considered to be terms of a categorical
theory.  We show then the equivalence of this meaning to the one of
$\mu$-terms defining $\mu$-bicomplete categories.  On the
combinatorial side, parity games can be considered as recognizers of
infinite objects in a natural way: they recognize the set of
deterministic winning strategies for a chosen player. The two
different meanings of parity games, the algebraic one and the
combinatorial one, are then shown to coincide if the category of sets
and functions is being considered.  This leads to the characterization
of functors denoted by $\mu$-terms in the category of sets: a
$\mu$-term is translated into a parity game and its denotation in this
category is the set of deterministic winning strategies for the chosen
player.


This result
supports the claim that the algebra of parity games is the one of
$\mu$-bicomplete categories and that the combinatorics of
$\mu$-bicomplete categories is the one of parity games, a claim which
is meant to emphasize two possible directions of research. One goes
from the algebra to the combinatorics: for example, this work should
provide a starting point for elaborating game semantics of programming
languages that implement both inductive and coinductive types.  The
other direction goes from the combinatorics to the algebra: we are
proposing an alternative algebraic interpretation of the alternation
between ``finitely many times'' and ``infinitely often'' which occurs
so often in the theory of automata recognizing infinite objects. The
alternation is usually analyzed by means of complete lattices, of
ordinals and of approximants of least fixed points that occur nested
within greatest fixed points.  Our interpretation requires induction
and coinduction, that is, initial algebras and final coalgebras of
functors which are natural generalizations of least and greatest fixed
points.  A particular motivation for developing this work has been the
possibility of describing transformations of winning strategies by
means of arrows definable in every $\mu$-bicomplete category.  We have
reported partial results on the structure of arrows of
$\mu$-bicomplete categories in \cite{sanlccp}.

On several occasions categories with similar completeness properties
have been proposed. For example, in \cite{MR93g:18005} a category
$\Cat{C}$ is defined to be algebraically complete if every unary
endofunctor has an initial algebra.  This requirement appears to be
too strong, since the only complete categories with this property turn
out to be the complete quasi-orders. It is possible to relax the
requirement and ask only a given class of functors of the form
$F:\prod_{i \in I}\Cat{C} \rTo \Cat{C}$ to be closed under
parameterized initial algebras.  This approach is the one proposed in
\cite{MR98j:18007} and leads to define $2$-iteration theories
\cite{MR2001m:18006}.  This is also our approach with the proviso that
we are interested in a specific class which is required to be closed
under parameterized final coalgebras as well. In \cite{MR93g:18005}
final coalgebras are considered too, but they are flattened into
initial algebras: an algebraically compact category is defined there
to be an algebraically complete category such that the inverse of
every initial algebra is also a final coalgebra of the same functor.

In these and other contexts the equational properties of categorical
fixed points have been studied. Our aim here is to see these
properties at work.  Our starting point will be the Beki\v{c}
property. In its simplest form it is an inductive method for showing
that a system of equations admits a unique solution: a sufficient
condition for the system of equations
$$
\begin{system}
  x & = & f(x,y) \\
  y & = & g(x,y)
\end{system}
$$
\vfill\eject\noindent
to admit a unique solution is that
the system
$$
\begin{system}
  x & = & f(x,y)
\end{system}
$$
admits a unique solution $x = \dg{f}(y)$ for each choice of $y$ and
that either of the two equivalent systems
$$
\begin{system}
  x & = & \dg{f}(y) \\
  y & = & g(x,y)
\end{system}
\;\;\;\;
\begin{system}
  y & = & g(\dg{f}(y),y)
\end{system}
$$
admits a unique solution. The analogy of the Beki\v{c} property
with Gaussian elimination has been pointed in
\cite{simpsonplotkin,AN01}. The equivalence between the last two
systems is moreover the root of our algebraic interpretation of parity
games.  We shall state a categorical version of the Beki\v{c}
property; the property stated above is recovered when considering that
a set is a poset with a discrete order, and that a poset is a category
with at most one arrow between any two objects: an initial algebra of
an endofunctor of a discrete category is nothing else but a unique
fixed point.  The Beki\v{c} property allows to prove the equivalence
between $\mu$-terms and systems of functorial equations. These seem to
be better suited than $\mu$-terms for an analysis that emphasizes the
operational aspects.  \vspace{2mm}

The paper is structured as follows. In section \ref{sec:notation} we
explain the notation, introduce the principal concepts and state the
Beki\v{c} property.  In section \ref{sec:mubicomplete} we define
categorical $\mu$-terms and $\mu$-bicomplete categories. We prove that
locally presentable categories are $\mu$-bicomplete. In section
\ref{sec:paritygames} we define parity games and their algebraic
interpretation. We show that it is possible to interpret every parity
game on a category if and only if the category is $\mu$-bicomplete, by
giving a translation of parity games into $\mu$-terms and vice-versa.
In section \ref{sec:inset} we prove that the algebraic interpretation
of a parity game in the category of sets is the set of winning
strategies for a chosen player.  We add some examples and applications
of the theory so far developed.  Finally, in section
\ref{sec:conclusions}, we add concluding remarks.


